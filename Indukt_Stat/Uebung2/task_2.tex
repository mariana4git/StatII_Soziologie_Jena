\documentclass[11pt]{article}
\pagestyle{empty}
%\usepackage[latin1]{inputenc}
\usepackage[utf8]{inputenc}
\usepackage{a4wide}
\usepackage{amsmath}
\usepackage{amssymb}
\usepackage{amsthm}
\usepackage{german}
\usepackage{multirow,array}
\usepackage{hyperref}
 \usepackage{graphicx}
%\usepackage{ipe}
%\input{thmstyle-ger}

\parindent0mm
\sloppy

% Basic data
\newcommand{\VORLESUNG}{Induktive Statistik für Soziologinnen und Soziologen}
\newcommand{\STAFF}{Mariana Nold}
\newcommand{\ASSIGNMENT}{2}
\newcommand{\HANDOUT}{Montag, den 13. November   2017}
\newcommand{\DELIVER}{Freitag 24. November 2017}
\newcommand{\PRACTICAL}[1]{\marginpar{\tiny {\bf Aufgabe \\ abgeben!} #1}}
\newcommand{\FAUFTRAG}[1]{\marginpar{\tiny {\bf selbst entdeckendes Verstehen} #1}}
\newcommand{\titel}{Grundlagen des statistischen Testens}
\newcommand{\startwert}{3}

% Arbitrary packages and settings

\newcommand{\N}{\mathbb{N}}
\newcommand{\floor}[1]{\lfloor{#1}\rfloor}
\newcommand{\ceil}[1]{\lceil{#1}\rceil}
\newcommand{\half}[1]{\frac{#1}{2}}
\newcommand{\punkte}[1]{{\small{ }(#1 Punkte)}}
\newcommand{\punkt}[1]{{\small{ }(#1 Punkt)}}

\newcommand{\aufgabe}[1]{\item{\bf #1}}
\newcommand{\hinweis}{{\em Hinweis}}

\begin{document}
% Document title

\begin{center}
\ASSIGNMENT{}. Aufgabenblatt vom \HANDOUT{} zur Vorlesung 
\vspace*{0.5cm}

{\Large \VORLESUNG{}}
%\PRACTICAL{}
(\STAFF{}) 


\vspace*{0.5cm}
{\textbf{Thema:} \titel{}\\}
\vspace*{0.2cm}

{\small Abgabe: \DELIVER{}}
\vspace*{1cm}
\end{center}

Wichtige Definitionen:
\begin{enumerate}
\item{Inferenzschluss}
\item{gerichtete Hypothese}
\item{ungerichtete Hypothese}
\item{Fehler 1. Art ($\alpha$-Fehler)}
\item{Fehler 2. Art ($\beta$-Fehler)}
\item{Teststatistik des Zwei-Stichproben t-Test}
\item{Ablauf des Zwei-Stichproben t-Test}
\item{kritischer Wert und Ablehnbereich}
\item{p-Wert}
%\item{Binäre Variable und  Bernoulliverteilung}
%\item{Binomialverteilung}
\end{enumerate}
\vspace{2cm}
\begin{enumerate}\addtocounter{enumi}{\startwert}


\aufgabe{Der Ablauf eines Hypothesentestes}\\ \punkte{10}  \PRACTICAL{}\\
Schildern Sie wie der Hypothesentest in der klassischen Inferenz abläuft. Gehen Sie dabei auf die Frage ein, was ein Inferenzschluss bedeutet. % Folie 43 in talk2
%http://www.crashkurs-statistik.de/vorgehen-bei-hypothesentests/

\aufgabe{Lesen Mädchen besser als Jungs: Doppelter t-Test und Welch-Test}\\
Im Sommer-Semester haben wir uns in der letzten Vorlesung am 19.06.2017 mit der Frage beschäftigt, ob Mädchen 
besser lesen als Jungen. Diese Fragestellung sollen Sie nun mit dem doppelten t-Test untersuchen, den Sie in der
Vorlesung kennengelernt haben.

\begin{itemize}
\item{Der  Index $j$ wird genutzt um die Gruppen zu identifizieren:
     \[
     j=\left\{\begin{array}{ll} 1, \: \text{``weiblich,''} \\
        2, \:  \text{``männlich''}\end{array}\right. 
  \]

}
\item{Es sei $y_{i,j}$ die Anzahl der Kompetenzpunkte im Bereich Lesen, der $i$-ten Person in Gruppe $j.$
\begin{itemize}
\item{So ist z. B. $y_{3,1}$ die Leseleistung der dritten Schülerin und}
\item{$y_{100,2}$ die Leseleistung des 100. Schülers.}
\end{itemize}
}
\item{Die mittlere Leseleistung der Schülerinnen ist $\bar{y}_{1},$
die mittlere Leseleistung der Jungen ist $\bar{y}_{2}.$ }
\item{Es gibt $n_{1}:=2462$ Schülerinnen und  $n_{2}:=2539$ Schüler im Datensatz.}
\end{itemize}

\begin{enumerate}
\item{Lesen Sie den Datensatz  \texttt{pisa\_2012\_schueler.dta} ein. Nutzen Sie den \texttt{STATA}-Befehl
\\ \texttt{. keep if cnt== ``DEU''} \\ um alle Daten von nicht-deutschen Schülerinnen und Schülern aus dem Speicher
zu nehmen. Geben sie den Befehl \\ \texttt{. browse} \\ ein um sich den Datensatz anzusehen.}
\item{Die Information über das Geschlecht ist in der Variable \texttt{st04q01} enthalten. Nutzen Sie den Befehl \\ \texttt{. tab st04q01}\\
um eine Häufigkeitstabelle zu erstellen.}
\item{Berechnen Sie für die Gruppe der Mädchen und Jungen jeweils die mittlere Anzahl der Lesepunkte und beschreiben Sie das Ergebnis
jeweils in einem Satz.
Nutzen Sie die dazu die Befehle \texttt{. summarize pv1read if st04q01==1} und \texttt{. summarize pv1read if st04q01==2} }
\item{Stellen Sie die Verteilung der Lesepunkte für Mädchen und Jungen in zwei Boxplots nebeneinander dar. Vergleichen
Sie die Boxplots in zwei bis drei Sätzen. Nutzen Sie den Befehl \\
\texttt{. graph box pv1read, over(st04q01)}.}
\item{Die Forschungshypothese lautet, dass Mädchen besser  lesen als Jungen. Legen Sie das Signifikanzniveau auf $5\%$ fest. Formulieren Sie diese 
Hypothese als statistische Hypothese und nutzen Sie den \texttt{STATA}-Befehl  
\texttt{. ttest pv1read, by(st04q01)} um den doppelten t-Test durchzuführen.
Kann die Forschungshypothese nachgewiesen werden?}
\item{Wenn man davon ausgehen kann, dass die Varianzen in den beiden Gruppen gleich
sind, dann verwendet man den doppelten t-Test. Wenn man nicht von der Varianzgleichheit ausgehen kann,
dann rechnet man den Welch-Test.  Rechnen Sie den Welch-Test und vergleichen Sie das 
Ergebnis mit dem des doppelten t-Test. (\texttt{. ttest pv1read, by(st04q01) welch}) }
%oneway pv1read st04q01   medage region
\end{enumerate}
%https://www.ssc.wisc.edu/sscc/pubs/sfs/sfs-ttest.htm
%sqrt(297.61)
%[1] 17.25138
\aufgabe{Die Füllmenge der Flaschen: Der einfach t-Test} \punkte{14}  \PRACTICAL{}\\

In einer Getränkefirma werden Flaschen von einer Maschine mit Mineralwasser gefüllt. Es ist bekannt,
dass die Füllmenge normalverteilt ist. Eine Mitarbeiterin vermutet, dass die Maschine zu wenig
Mineralwasser in die Flaschen füllt. Sie nimmt eine Stichprobe von $20$ Flaschen und bestimmt die
genaue Füllmenge.
\begin{enumerate}
\item{Lesen Sie den Datensatz \texttt{flaschen.dta} ein.}
\item{Nutzen Sie den Befehl \texttt{sum, detail} und lesen Sie den Mittelwert, den Median, das $10\%$-Quantil ($q_{0.1}$)
und das $90\%$-Quantil ($q_{0.9}$) ab. Interpretieren Sie das Intervall $(q_{0.1},q_{0.9})$ inhaltlich.} \punkte{5}
\item{Schlagen Sie ihre Formelsammlung auf S. 48 auf. Dort finden Sie dein einfachen t-Test. 
Legen Sie das Signifikanzniveau auf $1\%$ fest und Formulieren Sie die Hypothesen der Mitarbeiterin.}\punkte{2}
\item{Stellen Sie die Füllmenge mit Hilfe eines Boxplots dar und beschreiben Sie den Boxplot in zwei
bis drei Sätzen. (\texttt{. graph box Volume})}\punkte{2}
\item{Nutzen Sie den Befehlt \texttt{ttest Volume == 500} um den einfachen T-test zu rechnen.
Interpretieren Sie das Ergebnis inhaltlich. Muss die Maschine neu eingestellt werden?}
\item{Wie würde die Testentscheidung ausfallen, wenn das Signifikanzniveau bei $10\%$ liegen würde?}\punkte{1}
\item{Formulieren Sie den Fehler 2. Art inhaltlich. Was bedeutet dieser Fehler für die Getränke-Firma.}\punkte{2}
\item{Was könnte die Mitarbeiterin tun, um mehr Sicherheit im Bezug auf ihre Entscheidung zu haben? Begründen Sie 
ihre Antwort.}\punkte{2}
\item{Welche Abweichung von der gewünschten Füllmenge würde ihrer Meinung nach gerade noch zulässig sein?}
\end{enumerate}
% Beispiel mit den Flaschen
% http://www.instantr.com/2012/12/29/performing-a-one-sample-t-test-in-r/
% http://stats.seandolinar.com/one-sample-t-test-with-r-code/
% https://www.stata.com/manuals13/rttest.pdf
% http://www.r-tutor.com/elementary-statistics/type-2-errors/type-2-errors-two-tailed-test-population-mean-unknown-variance


\end{enumerate}
\end{document}
