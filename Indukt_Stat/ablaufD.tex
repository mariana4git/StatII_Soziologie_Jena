
\documentclass[a4paper,fontsize=12pt]{scrartcl}
\usepackage[utf8]{inputenc}
\usepackage{hyperref}
\usepackage[ngerman]{babel} 
\usepackage{textcomp} 
\usepackage[margin=1.55cm
  %,showframe% <- only to show the page layout
]{geometry}
\usepackage{rotating}
\usepackage{booktabs} 
\usepackage{lscape}
\usepackage{float}
%opening
\title{Deskriptive Statistik}
%\subtitle{Simulation und Auswertung der simulierten Daten in Übung und Tutorium mit Hilfe von \texttt{R}}
%\author{Mariana Nold}
\date{\vspace{-10ex}}
\begin{document}

\maketitle
\begin{enumerate}
\item{\textbf{Motivation: Wozu braucht man Statistik?}
  \begin{itemize}
    \item {Welche Geschichte erzählen die Daten: Deskriptive Statistik}
    \item {Vermutung über Signal und Rauschen: Explorative Statistik}
    \item {Ein Signal nachweisen: Induktive Statistik}
    \item {Transparenz ist sehr wichtig!}
  \end{itemize}
  }
  \item \textbf{Welche Aufgaben hat die deskriptive Statistik?}
  \begin{itemize}
     \item Merkmalstypen: Qualitative und quantitative Merkmale
     \item Skalen: Definition und Beispiele 
     \item Rohdaten: Merkmale, Merkmalsträger, Merkmalsausprägungen
     \item Definition: Absolute und relative Häufigkeit
     \item Definition: Univariate und multivariate Exploration
  \end{itemize}
  
 % \item{Was sind Rohdaten?
   
 % }
  \item{ \textbf{Wie analysiere ich qualitative Merkmale?}
    \begin{itemize}
     \item{Univariate Deskription und Exploration 
      \begin{itemize}
        \item{Lagemaße: Modus und Median}
        \item{Graphische Darstellung: Kreis- und Balkendiagramm}
      \end{itemize}
     } % Lagemaße, Kreis- und Balkendiagramm
     \item{Multivariate Deskription und Exploration} % Tabllen, interaktive Grafiken (Mondrian) % Mosaik-Plot
     \begin{itemize}
        \item{Kreuztabellen, bedingte Häufigkeit} % 
        \item{Graphische Darstellung: zweidimensionales Balkendiagramm,  Mosaik-Plot}
      \end{itemize}
    \end{itemize}
  }
    \item{\textbf{Univariate Deskription und Exploration von  quantitativen Merkmalen}
    \begin{itemize}
     \item{Lagemaße und Streuungsmaße: Modus, Median, Quantile und Mittelwert} % 
        \item{Spannweite, Interquartilsabstand, Varianz und Standardabweichung}
        \item{Graphische Darstellung: Boxplot und Histogramm}
 \end{itemize}
  }
  
     \item{\textbf{Multivariate Deskription und Exploration von  quantitativen Merkmalen}
    \begin{itemize}
    
         \item{Korrelation und zweidimensionale lineare Regression (deskriptiv)} % 
         \item{Graphische Darstellung: Streudiagramm und Regressionsgerade}
    \end{itemize}
  }
  \item{\textbf{Drittvariablenkontrolle}
    \begin{itemize}
      \item{Definition: Drittvariable bzw. Störfaktor}
      \item{Multivariate Regression (deskriptiv)}
    \end{itemize}
  }
  
  \item{\textbf{Varianzanalyse (deskriptiv)}
    \begin{itemize}
      \item{Vergleich von Mittelwerten in zwei Gruppen}
      \item{Vergleich von Mittelwerten in drei und mehr Gruppen}
    \end{itemize}
  }
  
  %\item{\textbf{Wiederholung}}
  
\end{enumerate}

% Aufgaben der deskrip. Statistik:

% ist es, die in den Daten einer Stichprobe enthaltene relevante Information in Tabellen, Grafiken und statistischen 
% % Maßzahlen übersichtlich und in einem der Fragestellung angemessenen Format zusammenzufassen.

% In der Drittvariablenkontrolle versucht man
% Effekte zwischen zwei Variablen unter
% Konstanthaltung (Aussschaltung) der
% anderen ins Modell aufgenommenen
% Variablen zu berechnen

\end{document}
