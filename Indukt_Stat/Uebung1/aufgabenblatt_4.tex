\documentclass[11pt]{article}
\pagestyle{empty}
%\usepackage[latin1]{inputenc}
\usepackage[utf8]{inputenc}
\usepackage{a4wide}
\usepackage{amsmath}
\usepackage{amssymb}
\usepackage{amsthm}
\usepackage{german}
\usepackage{multirow,array}
\usepackage{hyperref}
 \usepackage{graphicx}
%\usepackage{ipe}
%\input{thmstyle-ger}

\parindent0mm
\sloppy

% Basic data
\newcommand{\VORLESUNG}{Deskriptive Statistik für Soziologinnen und Soziologen}
\newcommand{\STAFF}{Mariana Nold}
\newcommand{\ASSIGNMENT}{4}
\newcommand{\HANDOUT}{Montag, den 22.Mai   2017}
\newcommand{\DELIVER}{bis Freitag, den 2. Juni 2017, Briefkasten des Instituts für Soziologie, in der Nähe der Cafeteria der Carls-Zeiss-Straße (In einem Umschlag, an mich adressiert) \textbf{oder} Mittwoch, Donnerstag und Freitag von 13-15 Uhr im Sekretariat von Frau Prof. Leuze, CZ-Straße 2, R286}
\newcommand{\PRACTICAL}[1]{\marginpar{\tiny {\bf Aufgabe \\ abgeben!} #1}}
\newcommand{\FAUFTRAG}[1]{\marginpar{\tiny {\bf selbst entdeckendes Verstehen} #1}}
\newcommand{\titel}{Bivariate Exploration von  quantitativen und qualitativen 
Merkmalen: Korrelation}
\newcommand{\startwert}{14}

% Arbitrary packages and settings

\newcommand{\N}{\mathbb{N}}
\newcommand{\floor}[1]{\lfloor{#1}\rfloor}
\newcommand{\ceil}[1]{\lceil{#1}\rceil}
\newcommand{\half}[1]{\frac{#1}{2}}
\newcommand{\punkte}[1]{{\small{ }(#1 Punkte)}}
\newcommand{\punkt}[1]{{\small{ }(#1 Punkt)}}

\newcommand{\aufgabe}[1]{\item{\bf #1}}
\newcommand{\hinweis}{{\em Hinweis}}

\begin{document}
% Document title

\begin{center}
\ASSIGNMENT{}. Aufgabenblatt vom \HANDOUT{} zur Vorlesung 
\vspace*{0.5cm}

%FT S149 + 156
{\Large \VORLESUNG{}}
%\PRACTICAL{}
(\STAFF{}) 


\vspace*{0.5cm}
{\textbf{Thema:} \titel{}\\}
\vspace*{0.2cm}

{\small Abgabe: \DELIVER{}}
\vspace*{1cm}
\end{center}

\begin{enumerate}\addtocounter{enumi}{\startwert}




\aufgabe{Wortschatz von Kindern} \punkte{20} \PRACTICAL{}\\
(in Anlehnung an: Fahrmeir et al, Statistik Der Weg zur Datenanalyse, S .151)\\
Bei fünf zufällig ausgewählten Kindern wurden der Wortschatz $X$ und die 
Körpergröße $Y$ in cm gemessen. Dabei erfolgte die Messung des Wortschatzes
über die Anzahl der verschiedenen Wörter, die die Kinder in einem Aufsatz über die 
Ergebnisse in ihren Sommerferien benutzten. Nehmen wir an,
wir hätten folgende Daten erhalten:

 \begin{table}[h]
 \centering 
\begin{tabular}{|r|r|r|r|r|r|}
  \hline
   Kind  $i$     &   $1$    & $2$   & $3$   & $4$   & $5$ \\ \hline
   Körpergröße $x_{i}$ & $37$ & $30$ & $20$ & $28$ & $35$  \\ \hline
   Wortschatz  $y_{i}$ & $130$& $112$ & $108$&  $114$ & $136$ \\ \hline
  \end{tabular}
 \caption{Der Wortschatz $X$ und die Körpergröße $Y$ in cm gemessen von $5$
 zufällig ausgewählten Kindern. \label{tab1}}
 \end{table}  
 
\begin{enumerate}
\item{Zeichnen Sie ein Streudiagramm.} \punkte{2}
\item{Erklären Sie an Hand dieses Beispiels was eine \punkte{4}
\begin{itemize}
\item{positive bzw. negative lineare Korrelation} %\punkte{3}
\item{positive bzw. negative monotone Korrelation} %\punkte{2}
\end{itemize}
inhaltlich bedeuten.} 
\item{Ist die folgende Aussage falsch oder richtig: Es ist im Allgemeinen möglich,
dass ein positiver linearer Zusammenhang vorliegt, aber kein positiver
monotoner Zusammenhang.} \punkte{2}
\item{Schreiben Sie die Tabelle der Ränge von $X$ und $Y,$
berechnen die den Rangkorrelationskoeffizient $r^{SP}_{XY}$ und interpretieren Sie
diesen Wert.}\punkte{4}
\item{Berechnen Sie nun  den Korrelationskoeffizienten nach Pearson $r_{XY}$
und interpretieren Sie diesen Wert.}\punkte{4}
 \begin{table}[h]
 \centering 
\begin{tabular}{|r|r|r|r|r|r|}
  \hline
  Kind   $i$     &   $1$    & $2$   & $3$   & $4$   & $5$ \\ \hline
  Körpergröße  $x_{i}$ & $37$ & $30$ & $20$ & $28$ & $35$  \\ \hline
  Wortschatz  $y_{i}$ & $130$& $112$ & $108$&  $114$ & $136$ \\ \hline
  Alter  $z_{i}$ & $12$& $7$ & $6$&  $7$ & $13$ \\ \hline
  \end{tabular}
 \caption{Der Wortschatz $X,$  die Körpergröße $Y$ in cm gemessen
 und das Alter $Z$ von $5$
 zufällig ausgewählten Kindern. \label{tab2}}
 \end{table}  
 \item{Die Tabelle \ref{tab2} enthält zusätzlich das Alter $Z$ der Kinder,
berechnen Sie jeweils $r_{YZ}$ und $r_{XZ}$ und interpretieren
Sie auch diese Werte.}\punkte{4}

\end{enumerate}


\aufgabe{Kreuztabellen interpretieren: Habilitationsdichte}   \punkte{11} \PRACTICAL{}\\

Die Habilitation ist die höchstrangige Hochschulprüfung in Deutschland durch Anfertigung einer wissenschaftlichen Arbeit.
In einer Untersuchung zur Habilitationsdichte an deutschen Hochschulen wurden u. a. die Merkmale Geschlecht und Habilitationsfach erhoben. In Tabelle 
\ref{tab4} ist - nach Fächern aufgeschlüsselt- zusammengefasst, wieviele Habilitationen im Jahre 2015 erfolgreich abgeschlossen wurden 
(Quelle: Statistisches Bundesamt) Hier stellt sich die Frage, ob die Habilitationsdichte in den einzelnen Fächern im Jahr 2015
geschlechtsspezifisch ist, d. h. man interessiert sich dafür, ob zwischen den Merkmalen Geschlecht (=:Y) und Habilitationsfach (=:X) ein Zusammenhang besteht.


   \[
   \text{Ausprägungen} \: X  \overset{\wedge}{=} \left\{\begin{array}{ll} % K, &  \text{\glqq keine Ausbildung\grqq} \\
         a_1, &  \text{ Geisteswissenschaften} \\
         a_2, &  \text{ Rechts-,Wirtschafts-,Sozialwiss.} \\
         a_3, &  \text{ Mathe u. Naturwiss.} \\
         a_4, &  \text{Human-,Gesundheitswiss.} \\
         a_5, &  \text{ übrige Fächer} \\
        \end{array}\right.
  \]
  
     \[
    \text{Ausprägungen} \:  Y  \overset{\wedge}{=} \left\{\begin{array}{ll}  
         b_1, &  \text{  Frauen} \\
         b_2, &  \text{ Männer } \\
        \end{array}\right.
  \]

\begin{table}[h]
\centering
\begin{tabular}{l l|c|c|c|c|c|c}
\multicolumn{2}{c}{}&\multicolumn{4}{c}{$X$}& &$\sum$\\
%\cline{3-7}
\multicolumn{2}{c}{}&$a_{1}$& $a_{2}$& $a_{3}$& $a_{4}$ & $a_{5}$&\\
\cline{3-7}
\multirow{2}{*}{$Y$}& $b_{1}$ & $77 $ & $62$ &  $ 66$ & $225$ & $32$&\\
\cline{2-7}
& $b_{2}$ & $159$ & $139$ &  $181$ & $571$ & $115$&\\
\cline{2-7}
& $\sum$ &  &  &   &  & &\\
%\caption{\label{tab4}}
\end{tabular}
\caption{Habilitationen im Jahre 2015 erfolgreich abgeschlossen wurden nach Fächern und Geschlecht aufgeschlüsselt\label{tab4} mit Randhäufigkeiten.}
\end{table}
\begin{enumerate}
\item{Ergänzen sie die fehlenden Randhäufigkeiten.} \punkt{1}
\item{Berechnen Sie die Randverteilungen.}\punkte{2}
\item{Wie hoch ist der Anteil der Frauen, die im Jahr $2015$ ein Habilitation abgeschlossen
haben.}\punkt{1}
\item{Wie hoch ist der Anteil an Habilitationen aus dem Fachbereich \glqq Mathematik
und Naturwissenschaften\grqq?}\punkt{1}
\item{Berechnen Sie die  bedingten Verteilungen gegeben dem Fachbereich
und interpretieren Sie das Ergebnis. } \punkte{3}
\item{Berechnen Sie die  bedingten Verteilungen gegeben das Geschlecht
und interpretieren Sie das Ergebnis. } \punkte{3}
\end{enumerate}

\newpage
\aufgabe{Die Korrelation mit STATA berechnen}


Das Streudiagramm \ref{fig1} zeigt die Spielbewertung eines neuen Spiels aufgetragen
auf der Ordinate und die Mathe-Punkte (hier simuliert, nicht aus den PISA-Daten)
von $300$ Schülerinnen bzw. Schülern. Sie finden den entsprechenden Datensatz
auf den Rechnern im Methoden-Labor im Ordner \texttt{Methoden/Statistik.}
% https://statistics.laerd.com/stata-tutorials/spearmans-correlation-using-stata.php
% https://statistics.laerd.com/stata-tutorials/pearsons-correlation-using-stata.php
%  \begin{figure}[ht]
% 	\centering
% 	      \includegraphics[width=0.75\textwidth]{scatter_curve2.pdf}
% 	      \caption{Spielfreude der Schülerinnen und Schüler versus Mathe-Score beruhend auf 
% 	      simulierten Daten. \label{fig1}}
% 	\end{figure}
 	\begin{enumerate}
 	\item{Öffnen Sie den Datensatz und geben Sie den Befehl \texttt{summarize} in das \texttt{command}-Fenster
 	ein. Interpretieren Sie die von \texttt{STATA} erzeugte Tabelle.}
 	\item{Geben Sie die Befehle \texttt{graph box x} und \texttt{graph box y} ein und interpretieren Sie
 	die entsprechenden Boxplots.}
 	\item{Erzeugen Sie mit dem Befehl \texttt{scatter y x} das Streudiagramm.}
 	\item{Berechnen Sie mit \texttt{pwcorr x y} den Korrelationskoeffizient nach Pearson.
 	Wie ändert sich das Ergebnis, wenn Sie den Befehl \texttt{pwcorr y x} eingeben.
 	Interpretieren Sie diese Veränderung inhaltlich.}
 	\item{Berechnen Sie mit Hilfe des Befehls \texttt{ spearman x y} den Wert des Rangkorrelationskoeffizienten.}
 	\end{enumerate}
%http://stackoverflow.com/questions/10711395/spearman-correlation-and-ties



\end{enumerate}

\end{document}
