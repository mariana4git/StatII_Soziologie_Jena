\documentclass[11pt]{article}
\pagestyle{empty}
%\usepackage[latin1]{inputenc}
\usepackage[utf8]{inputenc}
\usepackage{a4wide}
\usepackage{amsmath}
\usepackage{amssymb}
\usepackage{amsthm}
\usepackage{german}
\usepackage{multirow,array}
\usepackage{hyperref}
 \usepackage{graphicx}
%\usepackage{ipe}
%\input{thmstyle-ger}

\parindent0mm
\sloppy

% Basic data
\newcommand{\VORLESUNG}{Deskriptive Statistik für Soziologinnen und Soziologen}
\newcommand{\STAFF}{Mariana Nold}
\newcommand{\ASSIGNMENT}{4}
\newcommand{\HANDOUT}{Montag, den 22.Mai   2017}
\newcommand{\DELIVER}{keine Abgabe, wird in der Übung besprochen}
\newcommand{\PRACTICAL}[1]{\marginpar{\tiny {\bf Aufgabe \\ abgeben!} #1}}
\newcommand{\FAUFTRAG}[1]{\marginpar{\tiny {\bf selbst entdeckendes Verstehen} #1}}
\newcommand{\titel}{Bivariate Exploration von  quantitativen und qualitativen 
Merkmalen: Korrelation}
\newcommand{\startwert}{14}

% Arbitrary packages and settings

\newcommand{\N}{\mathbb{N}}
\newcommand{\floor}[1]{\lfloor{#1}\rfloor}
\newcommand{\ceil}[1]{\lceil{#1}\rceil}
\newcommand{\half}[1]{\frac{#1}{2}}
\newcommand{\punkte}[1]{{\small{ }(#1 Punkte)}}
\newcommand{\punkt}[1]{{\small{ }(#1 Punkt)}}

\newcommand{\aufgabe}[1]{\item{\bf #1}}
\newcommand{\hinweis}{{\em Hinweis}}

\begin{document}
% Document title

\begin{center}
\ASSIGNMENT{}. Übungsblatt vom \HANDOUT{} zur Vorlesung 
\vspace*{0.5cm}

{\Large \VORLESUNG{}}
%\PRACTICAL{}
(\STAFF{}) 


\vspace*{0.5cm}
{\textbf{Thema:} \titel{}\\}
\vspace*{0.2cm}

{\small Abgabe: \DELIVER{}}
\vspace*{1cm}
\end{center}

\begin{enumerate}\addtocounter{enumi}{\startwert}




\aufgabe{Standardisierung} % \PRACTICAL{}

Die folgende Grafik \ref{fig1} ist Ihnen aus der Vorlesung bekannt. Sie zeigt die Verteilung der Lese-Punkte der $5001$ deutschen Schülerinnen und Schüler
aus der Pisa-Studie von $2012.$ In grün ist
 eine mögliche Approximation aus der  Familie der Normalverteilungen eingezeichnet. 
 
 Die Lese-Leistung der Person $i$ ($i \in \{1,\dots, 5001\}$) wird mit $y_{i}$
bezeichnet. Der Mittelwert $\bar{y}$ ergibt sich zu $507.465$ Punkten. Die Standardabweichung $\hat{\sigma}_{Y}$ beträgt $91.263$ Punkte.

%\begin{figure}[ht]
% 	\centering
% 	      \includegraphics[width=0.60\textwidth]{dens_hist_lesen.pdf}
% 	      \caption{Histogramm und Modellverteilung der Lese-Punkte von 15-jährigen deutschen Schülerinnen und Schülern aus der Pisa-Studie $2012.$  \label{fig1}}
% 	\end{figure}

 \begin{enumerate}
 \item{Wie können Sie beruhend auf diesen Daten eine Vermutung über die Werte der Parameter $\mu$ und $\sigma$ der Normalverteilung gewinnen?}
 \item{Erklären Sie jeweils, wie sich die Form der Normalverteilung ändert, wenn man}
 \begin{itemize}
 \item{für festes $\sigma$ den Erwartungswert $\mu$ variiert. Interpretieren Sie die inhaltliche Bedeutung.}
 \item{für festes $\mu$ den Erwartungswert $\sigma$ variiert. Interpretieren Sie die inhaltliche Bedeutung.}
 \end{itemize}
 \item{Nutzen Sie die $68-95-99.7$-Regel um beruhend auf ihrer Modellverteilung ein Intervall zu schätzen, indem die Leseleistung von etwa $99.7\%$
 der Schülerinnen und Schüler liegt. Dieses Intervall soll symmetrisch um den vermuteten Erwartungswert $\mu$ liegen. Interpretieren Sie dieses Intervall.}
 \item{Berechnen Sie jetzt ein Intervall, so dass es die Leistungen der mittleren $68\%$ enthält und interpretieren Sie auch dieses Intervall.}
 \item{Die Lese-Punkte $Y$ sollen nun standardisiert werden.   Lesen Sie S. 80 in dem Buch
  \glqq Statistik- Eine Einführung für Sozialwissenschaftler \grqq von Ludwig-Mayerhofer, Liebeskind und Geißler (auf dt-workspace) und erklären Sie die Bedeutung der Standardisierung.
  Geben Sie die Berechnungsformel für die standardisierten Lese-Punkte $Z$ an.}
  \item{Die erste Person im Datensatz hat $y_{1}=475.001$ Lese-Punkte erreicht. Berechnen Sie ihre standardisierte Lese-Punktzahl $z_{1}.$}
  \item{Nehmen Sie an, es gibt noch andere Tests als die in der Pisa-Studie verwendeten Tests um die Lese-Leistung von $15$-jährigen
  zu beurteilen. Diese Tests haben eine andere maximale Punktzahl. Wie kann Ihnen das Prinzip der
  Standardisierung helfen, zu vergleichen, ob die unterschiedlichen Test inhaltlich zu einer vergleichbaren Bewertung kommen?}
\end{enumerate}


\aufgabe{Die durch die Ausgleichsgerade erklärte Streuung}

In der Vorlesung hatten wir über den (vermuteten) linearen Zusammenhang der Mathe- und Lese-Punkte gesprochen. Die Grafik \ref{fig2} zeigt für die ersten $300$
Schülerinnen und Schüler den Zusammenhang in einem Streudiagramm. 

Die Mathe-Leistung der Person $i$ ($i \in \{1,\dots,300\}$) wird mit $x_{i}$
bezeichnet. Der Mittelwert dieser Personengruppe $\bar{x}$ ergibt sich zu $538.92$ Punkten. Die Standardabweichung
dieser Personengruppe $\hat{\sigma}_{X}$ beträgt $101.636$ Punkte. 

Die Lese-Leistung der Person $i$ ($i \in \{1,\dots,300\}$) wird wieder  mit $y_{i}$
bezeichnet. Der Mittelwert dieser Personengruppe $\bar{y}$ ergibt sich zu $523.431$ Punkten. Die Standardabweichung dieser Personengruppe $\hat{\sigma}_{Y}$ beträgt $94.457$ Punkte.
Der Korrelationskoeffizient nach Pearson hat den Wert 0.889.
% Selber rechnen am Beispiel von S.219-222
%\begin{figure}[ht]
% 	\centering
% 	      \includegraphics[width=0.60\textwidth]{scatter_line.pdf}
% 	      \caption{Lese- und Mathe-Punkte der ersten 300 Schülerinnen und Schüler.\label{fig2}}
%\end{figure}

\begin{enumerate}
\item{Berechnen Sie beruhend auf dem Zusammenhang $$\hat{\beta}=\hat{\rho}\cdot \bigg(\frac{\hat{\sigma}_{Y}}{\hat{\sigma}_{X}}\bigg),$$
die Steigung der in der Grafik \ref{fig2} eingezeichneten Geraden und interpretieren Sie diese Steigung.}
\item{Um die Gerade festzulegen, muss neben der Steigung noch der Achsenabschnitt $\hat{\alpha}$ berechnet werden. Die verwendete 
Formel ist $$ \hat{\alpha}=\bar{y}-\hat{\beta}\bar{x}.$$ Berechnen und interpretieren Sie den Achsenabschnitt.}
\item{Die erste Person im Datensatz hat $x_{i}=475.001$ Lesepunkte und $y_{i}=443.53$ Mathe-Punkte. Berechnen Sie den durch
die Gerade vorhergesagten Wert für diese Person. Dieser Wert wird mit $\hat{y_{i}}$ bezeichnet.}
\item{Interpretieren Sie den Abstand $y_{i}-\hat{y_{i}}$ sowohl grafisch als auch inhaltlich.}
\item{Welcher Spezialfall ergibt sich, wenn für eine Erhebung $\sum_{i}^{300} |y_{i}-\hat{y_{i}}|=0$ gilt. }
% Wenn dieser Speziallfall gilt, würde die gesamte Streeung sum_{i=1}^{n} (y_{i}-\bar{y}) durch die Gerade erklärt werden
\end{enumerate}



\aufgabe{Monotone Funktionen}
\begin{enumerate}
\item{Finden Sie je ein Beispiel für eine streng monoton wachsende, eine schwach monoton wachsende,
eine streng monoton fallende und eine schwach monoton fallenden mathematische Funktion.}
\item{Überlegen Sie jeweils zu welchem Merkmauszusammenhang 
zwischen zwei Merkmalen $X$ und $Y$ (Streudiagramm) diese Funktion passen könnte.}
\end{enumerate}

\end{enumerate}
\end{document}
