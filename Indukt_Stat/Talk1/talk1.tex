% multi: https://texblog.org/2012/12/21/multi-column-and-multi-row-cells-in-latex-tables/

% Mit pdflatex mindestens 2mal uebersetzen und Ergebnis mit einem pdf-Viewer betrachten
%\documentclass{beamer}
% https://en.wikibooks.org/wiki/LaTeX/Colors
\documentclass[usenames,dvipsnames,handout]{beamer}
%\usepackage[latin1]{inputenc}
%\usepackage[ngerman]{babel}
\usepackage[utf8]{inputenc}
\usepackage[ngerman]{babel} 
\usepackage{color}
\usepackage{kpfonts}
\newcommand{\oldtextdied}{{\fontfamily{cmr}\selectfont\textdied}}
\newcommand{\oldtextborn}{{\fontfamily{cmr}\selectfont\textborn}}
\usepackage{multirow,array}
%\usepackage{multirow}
\usepackage{hyperref}
\usepackage{tikz}
\usetikzlibrary{shapes.geometric, arrows}
\usetikzlibrary{fit,arrows,calc,positioning}
% http://tex.stackexchange.com/questions/33231/how-to-change-the-color-of-a-block-within-a-custom-beamer-sty-theme-file
\usepackage{color}
\definecolor{mygreen}{cmyk}{0.82,0.11,1,0.25}
\usetheme[secheader]{Boadilla}
\newenvironment{variableblock}[3]{%
  \setbeamercolor{block body}{#2}
  \setbeamercolor{block title}{#3}
  \begin{block}{#1}}{\end{block}}


\begin{document}
\author[Dr. Mariana Nold]{Dr. Mariana Nold}
% \begin{center}
\institute[Institut für Soziologie]{ Institut für Soziologie,\\ Fakultät für Sozial- und Verhaltenswissenschaften,\\ Lehrstuhl für
 empirische Sozialforschung und Sozialstrukturanalyse}
% \end{center}
 \date{}
\title [Verhalten von Zufallsvariablen]{Stetige und diskrete 
parametrische  Wahrscheinlichkeitsverteilungen}
\date{23. Oktober 2017}
\begin{frame}
\maketitle

  \begin{figure}[ht]
 	\centering
 	      \includegraphics[width=0.15\textwidth]{index.jpeg}
 	\end{figure}
\end{frame} 

\begin{frame}
  \frametitle{Übersicht}
  \tableofcontents
\end{frame}



\begin{frame}{Ziel der heutigen Veranstaltung \dots}
ist es die folgenden Fragen beantworten zu können:
\begin{block}{Zielfragen für heute}
\begin{enumerate}
\item{Wie interpretiert man die Normalverteilung als stetige
parametrische Wahrscheinlichkeitsverteilung?}
\item{Was ist die Binomialverteilung?}
\item{Wie interpretiert man die Parameter der Binomialverteilung?}
\item{Was bedeuten die Aussagen:}
\begin{description}
\item{Die Zufallsvariable $X$ ist normalverteilt.}
\item{Die Zufallsvariable $Y$ ist binomialverteilt.}
\end{description}
\item{Welche Kenngrößen dienen der Beschreibung von Zufallsvariablen?}
\end{enumerate}
\end{block}
\end{frame}

\section{Erinnerung: Fiktive Befragung zum Taschengeld}



\begin{frame}{Erinnerung: Einfache Zufallsstichprobe und Inferenzschluss}
\begin{variableblock}{Definition: Einfache Zufallsstichprobe}{bg=Orchid!30,fg=black}{bg=Plum!30,fg=black}
Eine einfache Zufallsstichprobe liegt dann vor, wenn jede der möglichen auf diese Weise gezogenen Stichproben vom Umfang n aus einer Grundgesamtheit vom Umfang $N$ die gleiche Chance hat aufzutreten. 
\end{variableblock}
\begin{enumerate}
\item{Nur bei Realisierung einer Zufallsauswahl kann von einer Stichprobe
\textbf{mit kontrollierter Irrtumswahrscheinlichkeit} auf die zugrundeliegende
Grundgesamtheit zurückgeschlossen werden. }
\item{Dieser auch \textbf{Inferenzschluss} bezeichnete Rückschluss
von Eigenschaften der Stichprobe auf Eigenschaften der Grundgesamtheit ist
Gegenstand der induktiven Statistik.}
\item{Ein Inferenzschluss ist stets mit Unsicherheit behaftet. Dies ergibt sich
zwingend daraus, dass nur eine Teilinformation zu Verfügung steht.}
\end{enumerate}
\end{frame}

\begin{frame}{Erinnerung: Fiktive Befragung zum Taschengeld}

\begin{itemize}
\item{Annahme: Bei der fiktiven Stichprobe von $100$ Grundschulkindern handelt es sich um eine einfache Zufallsstichprobe.}
\item{Die nächsten Folien sind aus der letzten Vorlesung und zeigen das Ergebnis.}
\item{Wir wollen einen Inferenzschluss ziehen. Das heißt wir wollen eine Aussage machen, die sich auf alle
$N$ Grundschulkinder in Jena bezieht. }
\item{Wir gehen davon aus, dass sich die Normalverteilung eignet, um die Verteilung in der Grundgesamtheit
zu beschreiben.}
\item{Wir nennen die entsprechenden Parameter $\mu_{G}$ und $\sigma_{G}.$ }
\end{itemize}
\end{frame}

\begin{frame}{Grafische Darstellung: Empirische Verteilungsfunktion und beschreibende Verteilungsfunktion}
        \begin{figure}[ht]
 	\centering
 	      \includegraphics[width=0.55\textwidth]{taschengeld_theo.pdf}
 	\end{figure}
\end{frame}

\begin{frame}{Grafische Darstellung: Histogramm und deskriptive Normalverteilung}
        \begin{figure}[ht]
 	\centering
 	      \includegraphics[width=0.65\textwidth]{taschengeld_dens.pdf}
 	\end{figure}
\end{frame}


\begin{frame}{Erinnerung: Ergebnis der fiktiven Erhebung in Zahlen}
Wir haben die fiktive Erhebung zum Thema Taschengeld abgeschlossen und wollen herausfinden,
ob sich die Normalverteilung eignet um die Daten zu beschreiben.
\begin{itemize}
\item{Ist die beobachtetet Verteilung der Daten symmetrisch?}
\begin{itemize}
\item{Mittelwert: $\bar{x}=12.33,$ Median (= $50\%$ Quantil): $\tilde{x}=12,34$}\pause
\item{1. Quartil (= $25\%$ Quantil): $x_{0.25}=10.52$}\pause
\item{3. Quartil (= $75\%$ Quantil): $x_{0.75}=14.07$}\pause
\item{Minimum: 5.36 und Maximum: 19.2}\pause
\end{itemize}
\item{Sprechen die Lagemaße für eine symmetrische Verteilung?}\pause
\item{Grafische Verfahren werden genutzt um einen Eindruck von den Daten zu bekommen. Wir setzen $\mu \equiv \hat{\mu}=\bar{x}=12.33$ und $\sigma^{2} \equiv \hat{\sigma}^{2}=7.26.$}\pause
\item{Das Zeichen $\equiv$ bedeutet, dass wir für den Modellparameter einen bestimmten Wert einsetzen.}
\end{itemize}
%Lage- und Streuungsmaße (auch Quantile)
% Begriff Zufallsvariable Mittag S. 156, FT S. 226
%VORSICHT: mean und sd nicht echt sondern geschätzt!!!
\end{frame}


\begin{frame}{Die Normalverteilung }
\begin{itemize}
\item{Wenn wir eine symmetrische Verteilung beobachtet haben,
haben wir die Normalverteilung genutzt um die Daten zu approximieren}
\begin{itemize}
\item{Entweder indem wir die Dichte durch das Histogramm der Daten legen oder}
\item{indem wir die Verteilungsfunktion durch die empirische Verteilungsfunktion der Daten legen.}
\end{itemize}
\item{Wenn wir für die Parameter $\mu$ und $\sigma$ bestimmte Werte festlegen, sind damit sowohl
die Dichte als auch die Verteilungsfunktion bestimmt.}
\end{itemize}
\begin{block}{Dichte als auch die Verteilungsfunktion \dots}
sind zwei verschiedene Möglichkeiten die sich entsprechen. Wenn man eins von beiden festlegt,
ist auch das andere bestimmt.
\end{block}
\end{frame}

\begin{frame}{Die Parameter schätzen}
\begin{itemize}
\item{Wenn wir die Parameter $$\mu_{G}:=\frac{1}{N} \sum_{i=1}^{N} x_{i}$$ und 
$$\sigma_{G}:= \sqrt {\frac{1}{N} \sum_{i=1}^{N} ( x_{i}- \mu_{G})}$$ kennen, dann kennen
wir die Verteilung in der Grundgesamtheit.}
\item{Wir kennen diese Parameter nicht und nutzen Verfahren um diese Parameter zu schätzen.}
\item{Wichtig ist, dass wir die mit dieser Vorgehensweise verbundene Unsicherheit quantifizieren können.}
\end{itemize}
\end{frame}

\begin{frame}{Erinnerung: Zufallsvariable}
Ein ganz zentraler Begriff in diesem Semester ist der der Zufallsvariable:
\begin{variableblock}{Definition: Zufallsvariable }{bg=Orchid!30,fg=black}{bg=Plum!30,fg=black}
    	Eine Variable oder ein Merkmal $X,$ dessen Werte oder Ausprägungen die Ergebnisse eines 
    	Zufallsvorgangs sind, heißt Zufallsvariable $X.$ Die Zahl $x,$ die $X$ bei der Durchführung
    	eines Zufallsvorgangs annimmt, heißt Realisierung oder Wert von $X$
\end{variableblock}
Für unser Beispiel definiere ich die Zufallsvariable  $X:$
\begin{equation*}
X := \text{Höhe des Taschengeld von Grundschulkind aus Jena in  Euro}
\end{equation*}

\end{frame}

\begin{frame}{Eine typisches Aufgabe in der Statistik}
\begin{enumerate}
\item{Wir machen die Annahme das die Zufallsvariable $X$
\begin{equation*}
X = \text{Höhe des Taschengeld von Grundschulkind aus Jena in  Euro}
\end{equation*} in der Grundgesamtheit normalverteilt ist. Diese Verteilung wird in der Grundgesamtheit
festgelegt durch die Parameter $\mu_{G}$ und $\sigma_{G}.$ 
}
\item{Wir wollen diese Parameter durch Zahlen schätzen und so eine Aussage über die Grundgesamtheit machen.}
\item{Wir sagen: Wir schätzen die Parameter der Verteilung der Zufallsvariable $X.$ }
\end{enumerate}
\end{frame}
\section{Das Verhalten von Zufallsvariablen}
\subsection{Die Wahrscheinlichkeitsverteilung}% Mittag S. 158
\begin{frame}{Die Wahrscheinlichkeitsverteilung}

\begin{variableblock}{Definition: Wahrscheinlichkeitsverteilung}{bg=Orchid!30,fg=black}{bg=Plum!30,fg=black}
Ein Modell, welches das Verhalten einer Zufallsvariable vollständig beschreibt, nennt man Wahrscheinlichkeitsverteilung
oder kurz Verteilung der betreffenden Zufallsvariable.
\end{variableblock}
\begin{itemize}
\item{Wenn im obigen Beispiel $\mu_{G}$ und $\sigma_{G}$ bekannt sind, ist jede Wahrscheinlichkeit
bekannt. Eben das bedeutet es, dass das Verhalten der Zufallsvariable vollständig beschrieben ist. }
\item{Es ist z. B. möglich die Wahrscheinlichkeit zu berechnen, dass ein Kind höchstens zehn Euro Taschengeld
bekommt.}
\end{itemize}
\begin{block}{Zwei Parameter genügen}
Zur vollständigen Beschreibung des Verhaltens einer normalverteilten Zufallsvariablen $X$ genügt es,
 ihre beiden Parameter $\mu$ und $\sigma$ zu kennen.
\end{block}
\end{frame}



\begin{frame}{Das Verhalten einer Zufallsvariable}
\begin{block}{Verteilung einer Zufallsvariable}
Die Begriffe ``Verhalten einer Zufallsvariable'' und ``Verteilung einer Zufallsvariable'' sind 
gleichbedeutend. Sie legen das probabilistische Gesetz des entsprechenden Zufallsvorgangs 
fest.
\end{block}
Diesen wichtigen Zusammenhang werde ich an eignen Beispielen verständlicher machen.\pause
\begin{variableblock}{Definition: Zufallsvorgang }{bg=Orchid!30,fg=black}{bg=Plum!30,fg=black}
    	Ein Zufallsvorgang führt zu einem von mehreren, sich gegenseitig ausschließenden Ergebnissen. Es ist vor der Durchführung ungewiss, welches Ergebnis
    	tatsächlich eintreten wird.
\end{variableblock}\pause
\end{frame}


% Es bedeutet, jede Wahrscheinlichkeit angeben zu können.
%

\begin{frame}{Beispiel: Befragung nach dem Taschengeld}
\begin{itemize}
\item{Die Befragung von Kindern nach ihrem Taschengeld ist ein Zufallsvorgang.}\pause
\item{Wir bezeichnen diesen Vorgang mit einer Zufallsvariable $X.$}\pause
\item{Das Verhalten dieser Zufallsvariable wird beschrieben durch Wahrscheinlichkeiten mit
denen bestimmte Ereignisse im Zufallsvorgang eintreten.}\pause
\item{Wenn man die Verteilung in der Grundgesamtheit kennt und eine einfache Zufallsstichprobe zieht,
dann ist das Verhalten der Zufallsvariable $X$ vollständig bekannt.}\pause
\item{Für die obigen Daten, kenne ich die Verteilung und damit das Verhalten der Zufallsvariable $X.$}
\end{itemize}
\end{frame}
% http://onlinestatbook.com/2/sampling_distributions/samp_dist_mean.html
% https://www.mathsisfun.com/data/standard-normal-distribution.html
\begin{frame}{Was möchten Sie wissen?}
Wie hoch ist die Wahrscheinlichkeit, dass \dots
\begin{itemize}
\item{die ersten drei Kinder die befragt werden in der Summe weniger als $33$ Euro Taschengeld bekommen?}\pause
\item{$28.19\%$}\pause
\item{Die ersten zwei befragten Kinder jeweils höchstens $12$ Euro Taschengeld erhalten?}\pause
\item{$25\%$}
\item{Wie hoch ist die Wahrscheinlichkeit, dass der Mittelwert einer Stichprobe von $10$ Kindern
mindestens den Wert $10$ hat?}\pause
\item{$98.25\%$}
\end{itemize}\pause

\end{frame}

\begin{frame}{Die stetige Wahrscheinlichkeitsverteilung}
\begin{block}{Eine Zahl hat immer die Wahrscheinlichkeit Null}
Es ist eine Eigenschaft von stetigen Wahrscheinlichkeitsverteilungen, dass eine konkrete Zahl mit
Wahrscheinlichkeit Null auftritt.
\end{block}
\begin{itemize}
\item{Man kann eine Zahl beliebig genau angeben, wenn die Zahl Realisation einer stetigen Zufallsvariable ist.}\pause
\item{Daher macht es  bei stetigen Variablen keinen Sinn, nach der Wahrscheinlichkeit zu fragen, dass genau
ein bestimmter Wert angenommen wird.}\pause
\item{Man kann z. B. fragen, wie hoch die Wahrscheinlichkeit ist, dass ein Kinder gerundet auf ganze Euro $12$
Euro Taschengeld bekommt. Dann fragt man nach dem Intervall $[11.5,12.5).$}
\end{itemize}
\end{frame}


\subsection{Diskrete Zufallsvariable}

\begin{frame}{Stetige und diskrete Zufallsvariablen}% Mittag S. 156
Wenn man die Ausprägungen eines Merkmals als Ergebnis eines Zufallsvorgang interpretiert,
spricht man das Merkmal als Zufallsvariable an und die Ergebnisse des Zufallsvorgangs als Realisierungen
oder Werte
der entsprechenden Zufallsvariable.
\begin{itemize}
\item{Eine Zufallsvariable heißt stetig, wenn gilt: Wenn die Zufallsvariable die Werte $a$ und $b$ annehmen kann,
dann kann sie auch jeden Wert zwischen $a$ und $b$ annehmen.}\pause
\item{Bei einer diskreten Zufallsvariable ist die Anzahl der Werte abzählbar. Das heißt man kann sie den natürlichen Zahlen
zuordnen. Diese sind \{1,2,3,\dots\}}\pause
\item{Insbesondere ist jede Zufallsvariable mit endlich vielen Ausprägungen diskret.}
\end{itemize}
\end{frame}

\begin{frame}{Beispiele für diskrete Zufallsvariablen} 
\begin{itemize}
\item{Anzahl der Kinder die $2018$ geboren werden.}\pause
\item{Anzahl der SMS die Sie heute bekommen.}\pause
\item{Das Geschlecht einer befragten Person.}
\end{itemize}
Die bekanntesten diskreten Verteilungen sind die\pause
\begin{enumerate}
\item{Bernoulliverteilung}\pause
\item{Binomialverteilung}\pause
\item{Poissonverteilung}\pause
\end{enumerate}
Ich möchte Sie mit der Bernoulliverteilung und der Binomialverteilung vertraut machen.\\
Die Bernoulliverteilung ist ein Spezialfall der Binomialverteilung.
\end{frame}

%\begin{frame}{Das Verhalten von Zufallsvariablen am Beispiel der Klausureinsicht}
%\end{frame}
%
%\begin{frame}
%\begin{equation*}
%Y := \text{Anzahl der Studierenden, die zur Klausureinsicht kommen}
%\end{equation*}
%\end{frame}

\begin{frame}{Beispiel: Wie oft trifft ein Basketballer den Korb?}
\begin{itemize}
\item{Ein Basketballer hat zehn versuche auf einen Korb zu werfen.}
\item{Diesen Vorgang können wir als  Zufallsvorgang  interpretieren.
  \[
     Z_{i}=\left\{\begin{array}{ll} 1, & \text{Treffer bei $i.$ Wurf } \\
         0, &  \text{kein Treffer bei $i.$ Wurf}\end{array}\right.,
  \]
  wobei der Index $i$ die Zahlen von $1$ bis $10$ annehmen kann.
}
\item{$z_{1}=0,z_{2}=1,z_{3}=0$ bedeutet, dass von den ersten drei Würfen nur der zweite ein Treffer war.}
\end{itemize}
Die Zufallsvariable $Y$ ist gleich:
\begin{equation*}
Y := \text{Anzahl der Treffer bei $10$ Würfen}=\sum_{i=1}^{10} Z_{i}.
\end{equation*}
\end{frame}

\begin{frame}{Das Verhalten der Zufallsvariable $Y$}
Wenn wir das Verhalten der Zufallsvariable $Y$ kennen, können wir alle Fragen
über Ereignisse mit Bezug auf diese Variable beantworten:\\
Wie wahrscheinlich ist es, dass \dots
\begin{itemize}
\item{\dots kein einziger Treffer erzielt wird?}
\item{\dots genau ein Treffer erzielt wird?}
\item{\dots weniger als acht Treffer erzielt werden?}
\item{\dots mindestens drei Treffer erzielt werden?}\pause
\end{itemize}
Dieses Beispiel stammt aus dem Abitur 2008 in NRW und \dots
\end{frame}

\begin{frame}{Fiasko beim Zentralabitur 2008 in NRW}
\begin{itemize}
\item{In NRW gab es erhebliche Kritik an einer Aufgabe zur Wahrscheinlichkeitstheorie im Abitur 2008.}\pause
\item{Den Schülerinnen und Schülern wurde sogar angeboten, die Prüfung zu wiederholen.}\pause
\item{Damit Sie verstehen können, was passiert ist, möchte ich Ihnen zunächst die Aufgabenstellung
vorlesen und dann das Modell der Binomialverteilung zunächst am Beispiel der Werfens einer Münze 
einführen.}\pause
\item{Kann man das selbe Modell nutzen, um das Werfen einer Münze und das Werfen eines Basketballers auf einen Korb
zu modellieren?}\pause
\item{Man brauch eine Modellebene um Daten zu analysieren und eben bei der Wahl der Modellierung für diese Modelleben kann auch
viel schief gehen.}
\end{itemize}
\end{frame}

\begin{frame}{Was soll ein  Modell leisten?}
\begin{itemize}
\item{Die soziale Wirklichkeit wird von probabilistischen Gesetzen bestimmt.}\pause
\item{Die Verteilung der Zufallsvariable ist ein \textbf{ Modell}, welches das Verhalten der entsprechenden Zufallsvariable vollständig beschreibt.}\pause
\item{Es geht nicht darum, die Wirklichkeit in einem Modell abzubilden. }\pause
\end{itemize}
\begin{block}{All models are wrong, but some are useful (Zitat von G. Box)}
Now it would be very remarkable if any system existing in the real world could be exactly represented by any simple model.  \dots
For such a model there is no need to ask the question ``Is the model true?". If ``truth'' is to be the ``whole truth" the answer must be ``No". The only question of interest is ``\textbf{Is the model illuminating and useful?}".
\end{block}
\end{frame}
%https://en.wikipedia.org/wiki/All_models_are_wrong
\begin{frame}{Was kann ein Modell leisten?}
"The most that can be expected from any model is that it can supply a useful approximation to reality: \textbf{All models are wrong, some models are useful}"
%GeorgeEPBox.jpg
  \begin{figure}[ht]
 	\centering
 	      \includegraphics[width=0.25\textwidth]{GeorgeEPBox.jpg}
 	      \caption{George  Box, \oldtextborn 1919- \oldtextdied 2013}
 	\end{figure}
\end{frame}

%  \[
%     Z_{i}=\left\{\begin{array}{ll} 1, & \text{Treffer bei $i.$ Wurf } \\
%         0, &  \text{kein Treffer bei $i.$ Wurf}\end{array}\right.,
%  \]
%  wobei der Index $i$ die Zahlen von $1$ bis $10$ annehmen kann.
%}
%\item{$z_{1}=0,z_{2}=1,z_{3}=0$ bedeutet, dass von den ersten drei Würfen nur der zweite ein Treffer war.}
%\end{itemize}
%Die Zufallsvariable $Y$ ist gleich:
%\begin{equation*}
%Y := \text{Anzahl der Treffer bei $10$ Würfen}=\sum_{i=1}^{10} Z_{i}.
%\end{equation*}

\begin{frame}{Die Binomialverteilung am Beispiel Münzwurf}% Mittag S. 
Wir Beginnen mit dem Werfen einer Münze und lernen das Modell der Binomialverteilung kennen. 
\begin{itemize}
\item{Nehmen Sie eine Münze und Werfen Sie diese zehn mal.\\ Wir legen fest: Treffer=Zahl}
\item{Es sei 
  \[
     \tilde{Z}_{i}=\left\{\begin{array}{ll} 1, & \text{Treffer bei $i.$ Münzwurf } \\
         0, &  \text{kein Treffer bei $i.$ Münzwurf}\end{array}\right.,
  \]
}
\item{Notieren Sie das Ergebnis der Zufallsvariable 
\begin{equation*}
\tilde{Y} := \text{Anzahl der Treffer bei $10$ Münzwürfen}=\sum_{i=1}^{10} \tilde{Z}_{i}
\end{equation*}
und heben Sie den Zettel auf.
}
\end{itemize}
\end{frame}

\begin{frame}{Die Verteilung der Zufallsvariable $\tilde{Z}_{i}$}
Zwei zentrale Eigenschaften:
\begin{enumerate}
\item{Die Wahrscheinlichkeit für einen Treffer ist $50\%.$}
\item{Die einzelnen Münzwürfe sind unabhängig, d. h. wenn Sie beim ersten Wurf einen Treffer
hatten, hat es keinen Einfluss auf den zweiten Wurf.}\pause
\end{enumerate}
\begin{itemize}
\item{Mit Gesetzen der Wahrscheinlichkeitstheorie lässt sich damit die Verteilung von $\tilde{Y}$ herleiten. 
Man nennt sie Binomialverteilung.}
\item{Die Verteilung der Zufallsvariablen $\tilde{Z}_{i}$ ist jeweils identisch (also 
$\tilde{Z}_{1}$ hat die selbe Verteilung wie $\tilde{Z}_{2}$ und wie $\tilde{Z}_{3}$ \dots). Sie heißt Bernoulliverteilung.}
\end{itemize}
\end{frame}

\begin{frame}{Die Bernoulliverteilung} %FT S.228,229
\begin{variableblock}{Definition: Binäre Variable und  Bernoulliverteilung}{bg=Orchid!30,fg=black}{bg=Plum!30,fg=black}
Es sein ein Ereignis mit $A$ bezeichnet.
Die Zufallsvariable  $V$
  \[
     \tilde{V}=\left\{\begin{array}{ll} 1, & \text{falls $A$ eintritt } \\
         0, &  \text{falls $A$ nicht eintritt }\end{array}\right.,
  \] kann die Werte $\{0,1\}$ annehmen und indiziert, ob $A$ eintritt oder nicht. Sie heißt \textbf{binäre Variable.}\\
  Wenn die Wahrscheinlichkeit dass $A$ eintritt mit $p$ bezeichnet wird, also $\mathbb{P}(A)=p,$ so folgt
  \begin{equation*}
 \mathbb{P}(V=1) = p, \:\:\: \mathbb{P}(V=0) = 1-p.
\end{equation*}
Diese heißt Bernoulliverteilung. Man schreibt:
 $$ V \sim Be(p)$$
\end{variableblock}
\end{frame}



\begin{frame}{Die Binomialverteilung}
\begin{variableblock}{Definition: Binomialverteilung}{bg=Orchid!30,fg=black}{bg=Plum!30,fg=black}
Wenn man ein Zufallsvariable $U$ auffassen kann als die Summe von $n$ unabhängig und identisch Bernoulli-verteilten
Zufallsvariablen $V_{i},$ $i \in \{1, \dots, n\},$ dann nennt man die Verteilung dieser Zufallsvariable Binomialverteilung.
Man schreibt:
 $$ U \sim Bin(n,p)$$
\end{variableblock}\pause
In Formelsprache:\pause
  \begin{equation*}
V_{i} \sim Be(p), \: iid, \: i \in \: \{1, \dots n\}
\end{equation*}
  \begin{equation*}
U := \sum_{i=1}^{n} V_{i}.
\end{equation*}
 (iid = independent identically distributed)
\end{frame}

% Wahrscheinlichkeitsfunktion und verteilung (diskret u. stetig)
% kann man in Formeln angeben oder grafisch
% die Formeln brauchen wir nicht
\begin{frame}{Wahrscheinlichkeitsfunktion bzw. Dichtefunktion}
\begin{itemize}
\item{Man kann die Verteilung von stetigen und diskreten Zufallsvariablen in mathematischen Formeln
aufschreiben. }\pause
\item{Die Formeln sind ein Möglichkeit ein probabilistisches Gesetzt auszudrücken. Man nennt sie Dichte(funktion)
oder Wahrscheinlichkeitsfunktion.}\pause
\item{Sie haben bereits die Dichte der Normalverteilung kennengelernt. Allerdings nicht in Form
einer mathematischen Ausdrucks sondern grafisch.}\pause
\item{Wir brauchen um die Kernkompetenzen umsetzen zu können, diese mathematischen Ausdrücke nicht.}\pause
\item{Sie sollten allerdings wissen, dass es sie gibt und sie sollten die entsprechenden Grafiken interpretieren können.}
\end{itemize}
\end{frame}
%https://www.stat.berkeley.edu/~stark/Java/Html/BinHist.htm
%viel besser:
%https://www.geogebra.org/m/CmHJuJxs

\begin{frame}{Der 10-fache Münzwurf: Wie hoch ist die Wahrscheinlichkeit, dass \dots}
  \begin{figure}[ht]
 	\centering
 	      \includegraphics[width=0.55\textwidth]{muenze.pdf}
 	   %   \caption{George  Box, \oldtextborn 1919- \oldtextdied 2013}
 	\end{figure}
\end{frame}

\begin{frame}{Wie ist die relative Häufigkeit hier im Hörsaal?}
\begin{itemize}
\item{Bitte werfen Sie ihr Ergebnis in den Korb.}
\item{Ich kann das Histogramm der relativen Häufigkeit erstellen.}
\item{Es sollte der Binomialverteilung ähnlich sein. Der Vergleich erfolgt in er Übung.}
\item{Bitte nutzen Sie den folgenden Link: \url{https://www.geogebra.org/m/CmHJuJxs}\\
um die Binomialverteilung kennen zu lernen.}
\end{itemize}
\end{frame}

\begin{frame}{Für welche realen Mechanismen ist die Binomialverteilung ein gutes Modell?}
\begin{itemize}
\item{Es gibt zwei Parameter $n$ und $p.$}\pause
\item{Man nennt $n$ die Anzahl der Versuche und $p$ die (Treffer-)Wahrscheinlichkeit.}\pause
\item{Man stellt sich $n$ unabhängige binäre Variablen vor, die summiert werden.}\pause
\item{Beispiele:
\begin{itemize}
\item{Wenn man $20$ Personen (jeweils einzeln) fragt, ob sie die Grünen bei einer Bundestagswahl wählen würden.
Wie ist $p$ hier zu interpretieren?}\pause
\item{p: Anteil der Personen, die die Grünen wählen würden.}
\item{Treffer wenn eine Person $10$ mal auf einen Basketballkorb wirft und man annimmt,
sie trifft mit $20\%$ und zwar unabhängig.}\pause
\end{itemize}
}
\end{itemize}
\end{frame}
\subsection{Aufgabenstellung im Mathe-Abi 2008: Wo lag das Problem?}
%\subsection{Die Aufgabenstellung}
\begin{frame}{Die Aufgabenstellung: }%Mittag S.
Wir wollen die Aufgabe nicht lösen, sondern überlegen, ob die Binomialverteilung hier anwendbar ist.\\ \pause
\begin{block}{Die Aufgabenstellung: E-Book von H.-J. Mittag S. 175}
Der deutsche Basketball-Profi Dirk Nowitzki spielte in der amerikanischen Profiliga beim Club Dallas Mavericks. In der Saison 2006/07
erzielte er bei Freiwürfen eine Trefferquote von $90.4\%.$ Berechnen Sie die Wahrscheinlichkeit dafür, dass er
\begin{itemize}
\item[(1)]{genau $8$ Treffer bei $10$ Versuchen erzielt,}
\item[(2)]{höchstens $8$ Treffer bei $10$ Versuchen erzielt,}
\item[(3)]{\dots}
\end{itemize}
\end{block}
\end{frame}

\begin{frame}{Einige Kritikpunkte an der Aufgabenstellung}
\begin{itemize}

\item{Die Analogie zwischen Freiwurf und Münzwurf liegt auf der Hand: Es sind binäre Merkmale}\pause
\item{Identische Verteilung: Beim Münzwurf bleibt die Wahrscheinlichkeit dafür Zahl zu werfen, von Wurf zu Wurf konstant. Trifft das auch
auf den Freiwurf beim Basketball zu?}\pause
\item{Beim Münzwurf sind die Würfe unabhängig, gilt das auch beim Freiwurf?}\pause
\item{Beim Münzwurf lässt sich die Trefferwahrscheinlichkeit von $0.5$ aus physikalischen Gründen als beständig ansehen.
Gilt das auch für die Trefferquote von $90.4\%$ ?}
%\item{Der Teil (3) der Aufgabenstellung ist unvollständig und daher \colorbox{yellow!20}{nicht lösbar.} Hier fehlt die Anzahl der
%Versuche $n.$}
\end{itemize}

\end{frame}

\begin{frame}{Welches Modell eignet sich?}
\begin{description}
\item{An der Aufgabe lässt sich verdeutlichen, wie wichtig es ist, zwischen empirischen Befunden (Datenebene) und
Modellansätzen zur approximativen Beschreibung solcher Befunde (Modellebene) zu unterscheiden.}\pause
\item{Es wird auch deutlich: Es ist ein Unterschied, ob man Daten die vorliegen beschreibt, oder ob man beruhend auf 
diesen Daten eine Vorhersage machen möchte, für zukünftige Ereignisse.}\pause
\item{Man kann darüber diskutieren, ob sich das Modell der Binomialverteilung hier als nützlich erweist. (" All models are wrong, some models are useful")}\pause
%\item{Man kann diese Frage nicht klar mit ja oder nein beantworten.}
%\item{Ein Modell bedeutet immer einen Kompromiss zu machen zwischen einer komplexen Wirklichkeit und dem Wunsch 
%probabilistische Gesetzmäßigkeiten dieser Wirklichkeit entdecken zu können.}
\end{description}
\end{frame}

\begin{frame}{Daten- und Modellebene}% Mittag S. 158
\begin{description}
\item{Zwischen empirischen Verteilungen von Merkmalen (Häufigkeitsverteilungen) und theoretischen
Verteilungen von Zufallsvariablen gibt es zentrale Analogien.}\pause
\item{Wichtig ist die Unterscheidung beider Konzepte, also Daten- und Modellebene.}\pause
\item{Verteilungen von Zufallsvariablen sind als Modelle zu verstehen, die oft gut geeignet sind, \textbf{Strukturen und Gesetzmäßigkeiten,}
die großen Datenmengen zu Grunde liegen können, \textbf{zu approximieren.}}
\end{description}
\end{frame}
\section{Kenngrößen  einer Zufallsvariable}

\begin{frame}{Empirische Verteilung und theoretische Verteilung}
       
       \begin{figure}[ht]
 	\centering
 	      \includegraphics[width=0.55\textwidth]{taschengeld_theo.pdf}
 	\end{figure}
\end{frame}



\begin{frame}{Kenngrößen für Zufallsvariablen}% Mittag S. 158
Wie bei empirischen Verteilungen lassen sich auch bei theoretischen Verteilungen Kenngrößen angeben, die das \textbf{Zentrum der Verteilung}
beschreiben oder die \textbf{Variabilität der Zufallsvariable,} die dieser Verteilung folgt.
\begin{itemize}
%\item{ }
\item{Im letzten Semester hatten wir in der deskriptiven Statistik Lage- und Streuungsmaße verwendet, um empirische Verteilungen zu beschreiben. }
\item{Wen man mit Daten arbeitet beschreibt man die empirische Verteilung.}
\item{Wenn man mit Modellen für Zufallsvariablen arbeitet, beschreibt man die (theoretische) Verteilung.}
%\item{Wenn wir die theoretische Verteilung beschreiben, wollen wir von Lage- und Streuungs\textbf{parametern} sprechen. Bei der empirischen
%Verteilung von Lage- und Streuungs\textbf{maßen}. }
\end{itemize}
\end{frame}

\begin{frame}{Wiederholung: Lagemaße in der deskriptiven Statistik}
 \begin{table}[ht]
\centering
%\caption{Was darf ich mit den Ausprägungen tun?}
\begin{center}
\caption{Messniveau und Lagemaße}
\begin{tabular}{c|c|c|c|c}
	\hline
	\multicolumn{5}{c}{Zulässige Lagemaße}\\ \hline
	Skala& Modus & Median & Quantile & Mittelwert\\\hline
	Nominals.    & \colorbox{green!30}{ja} & \colorbox{red!30}{nein} & \colorbox{red!30}{nein} & \colorbox{red!30}{nein} \\
	Ordinals.    & \colorbox{green!30}{ja} & \colorbox{green!30}{ja} & \colorbox{green!30}{ja} & \colorbox{orange!30}{jein\footnote{Eigentlich nein, aber es gibt Merkmale, die sich als ordinal oder intervallskaliert auffassen lassen.}} \\ \hline
	Intervalls.  & \colorbox{green!30}{ja} & \colorbox{green!30}{ja} & \colorbox{green!30}{ja} & \colorbox{green!30}{ja}\\
	Verhältniss. & \colorbox{green!30}{ja} & \colorbox{green!30}{ja} & \colorbox{green!30}{ja} & \colorbox{green!30}{ja} \\
	\hline
\end{tabular}
\end{center}
\label{tab:multicol0}
\end{table}
\end{frame}

\begin{frame}{Wiederholung: Streuungsmaße in der deskriptiven Statistik}
 \begin{table}[ht]
\centering
%\caption{Was darf ich mit den Ausprägungen tun?}
\begin{center}
\caption{Messniveau und Streuungsmaße}
\begin{tabular}{c|c|c|c|c}
	\hline
	\multicolumn{5}{c}{Zulässige Streuungsmaße}\\ \hline
	Skala& Spannweite & Interquartilsabst.& Varianz & Standardab.\\\hline
	Nominals.    & \colorbox{red!30}{nein} & \colorbox{red!30}{nein} & \colorbox{red!30}{nein} & \colorbox{red!30}{nein} \\
	Ordinals.    & \colorbox{green!30}{ja} & \colorbox{green!30}{ja} & \colorbox{red!30}{nein} & \colorbox{red!30}{nein} \\ \hline
	Intervalls.  & \colorbox{green!30}{ja} & \colorbox{green!30}{ja} & \colorbox{green!30}{ja} & \colorbox{green!30}{ja}\\
	Verhältniss. & \colorbox{green!30}{ja} & \colorbox{green!30}{ja} & \colorbox{green!30}{ja} & \colorbox{green!30}{ja} \\
	\hline
\end{tabular}
\end{center}
\label{tab:multicol0}
\end{table}
\end{frame}

\begin{frame}{Lagemaße und Streuungsmaße einer Zufallsvariable}
Häufig verwendete  Lagemaße der Verteilung einer Zufallsvariable sind:
\begin{itemize}
\item{der Erwartungswert}
\item{und die (theoretischen) Quantile}
\end{itemize}
Häufig verwendete Streuungsmaße der Verteilung einer Zufallsvariable sind:
\begin{itemize}
\item{die (theoretische) Varianz}
\item{und ihre Wurzel die (theoretischen) Standardabweichung}
\end{itemize}
\end{frame}

\begin{frame}{Erwartungswert}
Der Erwartungswert einer Zufallsvariable wird gebildet indem man die Werte die die Zufallsvariable annehmen kann mit der entsprechenden
Wahrscheinlichkeit gewichtet.
\begin{description}
\item{Bei stetigen Zufallsvariablen entspricht das mathematisch einer Integration.}\pause
\item{Bei diskreten Zufallsvariablen ist die Berechnung sehr ähnlich zur Berechnung des Mittelwertes. Man gewichtet die Ausprägung mit der Wahrscheinlichkeit.}\pause
\item{Für das Beispiel 10-facher Münzwurf werden wir uns diese Analogie in der Übung ansehen.}
\end{description}
\end{frame}

\begin{frame}{Erwartungswert und Varianz der Normal- und Binomialverteilung}
\begin{description}
\item{Der Erwartungswert der Normalverteilung entspricht dem Parameter $\mu.$ Die Standardabweichung ist $\sigma,$
die Varianz ist $\sigma^{2}.$}
\item{Die Normalverteilung ist durch Angabe von $\mu$ und $\sigma$ eindeutig bestimmt.}
\item{Der Erwartungswert der Binomialverteilung entspricht ist $n \cdot p.$ Die Standardabweichung ist $n \cdot p \cdot (1-p).$
}
\end{description}
\end{frame}

\begin{frame}{Literatur für das Arbeiten mit \texttt{STATA}}
Datenanalyse mit Stata: allgemeine Konzepte der Datenanalyse und ihre praktische Anwendung,
von Ulrich Kohler, Frauke Kreuter
\end{frame}
\end{document}

