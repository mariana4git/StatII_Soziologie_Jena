% multi: https://texblog.org/2012/12/21/multi-column-and-multi-row-cells-in-latex-tables/

% Mit pdflatex mindestens 2mal uebersetzen und Ergebnis mit einem pdf-Viewer betrachten
%\documentclass{beamer}
% https://en.wikibooks.org/wiki/LaTeX/Colors
\documentclass[usenames,dvipsnames,handout]{beamer}
%\usepackage[latin1]{inputenc}
%\usepackage[ngerman]{babel}
\usepackage[utf8]{inputenc}
\usepackage[ngerman]{babel} 
\usepackage{color}
\usepackage{multirow,array}
%\usepackage{multirow}
\usepackage{hyperref}
\usepackage{tikz}
\usetikzlibrary{shapes.geometric, arrows}
\usetikzlibrary{fit,arrows,calc,positioning}
% http://tex.stackexchange.com/questions/33231/how-to-change-the-color-of-a-block-within-a-custom-beamer-sty-theme-file
\usepackage{color}
\definecolor{mygreen}{cmyk}{0.82,0.11,1,0.25}
\usetheme[secheader]{Boadilla}
\newenvironment{variableblock}[3]{%
  \setbeamercolor{block body}{#2}
  \setbeamercolor{block title}{#3}
  \begin{block}{#1}}{\end{block}}


\begin{document}
\author[Dr. Mariana Nold]{Dr. Mariana Nold}
% \begin{center}
\institute[Institut für Soziologie]{ Institut für Soziologie,\\ Fakultät für Sozial- und Verhaltenswissenschaften,\\ Lehrstuhl für
 empirische Sozialforschung und Sozialstrukturanalyse}
% \end{center}
 \date{}
\title [Deskriptive und Induktive Statistik]{Stetige und diskrete 
parametrische  Wahrscheinlichkeitsverteilungen}
\date{16. Oktober 2017}
\begin{frame}
\maketitle

  \begin{figure}[ht]
 	\centering
 	      \includegraphics[width=0.15\textwidth]{index.jpeg}
 	\end{figure}
\end{frame} 

\begin{frame}
  \frametitle{Übersicht}
  \tableofcontents
\end{frame}

\section{Themen und Ziele}






\begin{frame}{Ziel der heutigen Veranstaltung \dots}
ist es die folgenden Fragen beantworten zu können:
\begin{block}{Zielfragen für heute}
\begin{enumerate}
\item{Welche Intention hat die deskriptive Statistik?}
\item{Welche Intention hat die induktive Statistik?}
\item{Was ist ein Zufallsvorgang?}
\item{Was ist eine einfache Zufallsstichprobe?}
\item{Was ist die Normalverteilung und wie kann man sie nutzen,
um einen Datensatz zu beschreiben?}
\item{Was versteht man unter einem Inferenzschluss?}
\end{enumerate}
\end{block}
\end{frame}

\end{document}

