% multi: https://texblog.org/2012/12/21/multi-column-and-multi-row-cells-in-latex-tables/

% Mit pdflatex mindestens 2mal uebersetzen und Ergebnis mit einem pdf-Viewer betrachten
%\documentclass{beamer}
% https://en.wikibooks.org/wiki/LaTeX/Colors
\documentclass[usenames,dvipsnames,handout]{beamer}
%\usepackage[latin1]{inputenc}
%\usepackage[ngerman]{babel}
\usepackage[utf8]{inputenc}
\usepackage[ngerman]{babel} 
\usepackage{color}
\usepackage{multirow,array}
%\usepackage{multirow}
\usepackage{hyperref}
\usepackage{tikz}
\usetikzlibrary{shapes.geometric, arrows}
\usetikzlibrary{fit,arrows,calc,positioning}
% http://tex.stackexchange.com/questions/33231/how-to-change-the-color-of-a-block-within-a-custom-beamer-sty-theme-file
\usepackage{color}
\definecolor{mygreen}{cmyk}{0.82,0.11,1,0.25}
\usetheme[secheader]{Boadilla}
\newenvironment{variableblock}[3]{%
  \setbeamercolor{block body}{#2}
  \setbeamercolor{block title}{#3}
  \begin{block}{#1}}{\end{block}}


\begin{document}
\author[Dr. Mariana Nold]{Dr. Mariana Nold}
% \begin{center}
\institute[Institut für Soziologie]{ Institut für Soziologie,\\ Fakultät für Sozial- und Verhaltenswissenschaften,\\ Lehrstuhl für
 empirische Sozialforschung und Sozialstrukturanalyse}
% \end{center}
 \date{}
\title [Punktschätzer und Konfidenzintervalle ]{Punktschätzer und Konfidenzintervalle }
\date{27. November 2017}
\begin{frame}
\maketitle

  \begin{figure}[ht]
 	\centering
 	      \includegraphics[width=0.15\textwidth]{index.jpeg}
 	\end{figure}
\end{frame} 

\begin{frame}
  \frametitle{Übersicht}
  \tableofcontents
\end{frame}

\section{Themen und Ziele}




%http://www.statmethods.net/stats/power.html

\begin{frame}{Ziel der heutigen Veranstaltung \dots}
ist es die folgenden Fragen beantworten zu können:
\begin{block}{Zielfragen für heute}
\begin{enumerate}
\item{Welche Probleme treten auf, wenn man den Fehler 2. Art nicht berücksichtigt?}
\item{Wie ist die Power eines Tests definiert?}
\item{Ziel 3}
\item{Ziel 4}
\item{Ziel 5}
\item{Ziel 6}
\end{enumerate}
\end{block}
\end{frame}
\section{Stichprobenumfang und Power}
\begin{frame}{Die Füllmenge der Flaschen}
\begin{itemize}
\item{In der Aufgabe 6 des letzten Aufgabenblattes ging es um die Frage, ob eine Maschine die 
Mineralwasserflaschen befüllt  neu eingestellt werden muss.}
\item{Die Maschine soll exakt
$500$ ml pro Flasche abfüllen. Nehmen wir an, der wahre Mittelwert der Füllmenge
ist $\mu$ und die wahre Standardabweichung $\sigma.$ Beide Parameter sind
unbekannt. Der Stichprobenumfang ist $n.$}
\item{Wir wollen uns heute zunächst mit dem Fehler 2. Art beschäftigen.  Also mit dem Fehler eine unwahre 
Nullhypothese nicht zu erkennen und beizubehalten.}
\end{itemize}

%\begin{frame}{Wie falsch darf die Maschine arbeiten?}
%%\begin{itemize}
%%\item{Die Toleranz wird auf $20$ ml in beide Richtungen festgelegt. Das bedeutet:
%%Eine Akzeptable Füllmenge liegt im Intervall $(480,520).$}
%%%\item{Da eine Abweichung der Füllmenge sowohl nach oben, als auch nach unten
%%%zu Problemen führt, arbeiten mir mit der ungerichteten Hypothese $\mu=\mu_{0}=500$}
%%\end{itemize}
%\end{frame}


\end{document}

