% multi: https://texblog.org/2012/12/21/multi-column-and-multi-row-cells-in-latex-tables/

% Mit pdflatex mindestens 2mal uebersetzen und Ergebnis mit einem pdf-Viewer betrachten
%\documentclass{beamer}
% https://en.wikibooks.org/wiki/LaTeX/Colors
\documentclass[usenames,dvipsnames,handout]{beamer}
%\usepackage[latin1]{inputenc}
%\usepackage[ngerman]{babel}
\usepackage[utf8]{inputenc}
\usepackage[ngerman]{babel} 
\usepackage{color}
\usepackage{multirow,array}
%\usepackage{multirow}
\usepackage{hyperref}
\usepackage{tikz}
\usetikzlibrary{shapes.geometric, arrows}
\usetikzlibrary{fit,arrows,calc,positioning}
% http://tex.stackexchange.com/questions/33231/how-to-change-the-color-of-a-block-within-a-custom-beamer-sty-theme-file
\usepackage{color}
\definecolor{mygreen}{cmyk}{0.82,0.11,1,0.25}
\usetheme[secheader]{Boadilla}
\newenvironment{variableblock}[3]{%
  \setbeamercolor{block body}{#2}
  \setbeamercolor{block title}{#3}
  \begin{block}{#1}}{\end{block}}


\begin{document}
\author[Dr. Mariana Nold]{Dr. Mariana Nold}
% \begin{center}
\institute[Institut für Soziologie]{ Institut für Soziologie,\\ Fakultät für Sozial- und Verhaltenswissenschaften,\\ Lehrstuhl für
 empirische Sozialforschung und Sozialstrukturanalyse}
% \end{center}
 \date{}
\title [Deskriptive und Induktive Statistik]{Der Zufallsvorgang und der Inferenzschluss}
\date{16. Oktober 2017}
\begin{frame}
\maketitle

  \begin{figure}[ht]
 	\centering
 	      \includegraphics[width=0.15\textwidth]{index.jpeg}
 	\end{figure}
\end{frame} 

\begin{frame}
  \frametitle{Übersicht}
  \tableofcontents
\end{frame}

\section{Themen und Ziele }

\begin{frame}{Kernkompetenzen}
Folgende Kernkompetenzen sollen Sie im Laufe des Semesters aufbauen:
\begin{block}{Statistische Methoden}
\begin{enumerate}
\item{Daten selbst analysieren: Thüringen-Monitor 2015 und Pisa-Daten mit \texttt{STATA}}
\item{Statistische Informationen nutzen und sachadäquat interpretieren: \texttt{STATA}-Output und Information aus Presse oder Fachliteratur }
\item{Statistische Ergebnisse verständlich kommunizieren, wesentliche Aussagen in eigenen Worten ausdrücken können}\\
\vspace{0.5cm}
\textbf{Voraussetzung dafür:}
\begin{itemize}
\item{Zentrale Definition kennen und verstehen}
\item{Den Umgang mit \texttt{STATA} üben}
\end{itemize}
\end{enumerate}
\end{block}
\end{frame}

\begin{frame}{Ziel der heutigen Veranstaltung \dots}
ist es die folgenden Fragen beantworten zu können:
\begin{block}{Zielfragen für heute}
\begin{enumerate}
\item{Welche Intention hat die deskriptive Statistik?}
\item{Welche Intention hat die induktive Statistik?}
\item{Was ist ein Zufallsvorgang?}
\item{Was ist eine einfache Zufallsstichprobe?}
\item{Was ist die Normalverteilung und wie kann man sie nutzen,
um einen Datensatz zu beschreiben?}
\item{Was versteht man unter einem Inferenzschluss?}
\end{enumerate}
\end{block}
\end{frame}



\begin{frame}{Thematische Schwerpunkte in diesem Semester}
Es gibt in diesem Semester drei thematische Schwerpunkte:
\begin{enumerate}
\item{\colorbox{blue!10}{Probabilistische Gesetze:   }
\begin{itemize}
\item{Was ist eine Zufallsvorgang bzw. eine Zufallsvariable?}
\item{Was ist eine parametrische  Wahrscheinlichkeitsverteilung?}
\end{itemize}
}
\item{\colorbox{green!40}{Testen und Schätzen:} 
\begin{itemize}
\item{Wie kommt man zu einer Punktschätzung?}
\item{Was ist ein Konfidenzintervall?}
\item{Auf welchen Prinzipien beruht das Konzept des Hypothesentests?}
\end{itemize}
}
\item{ \colorbox{violet!40}{Regression und Varianzanalyse}
\begin{itemize}
\item{Wie interpretiert man die Regressionskoeffizienten im linearen Regressionsmodell?}
\item{Wie nutzt man die Varianzanalyse zum Mittelwertvergleich bei mehr als zwei Gruppen?}
\end{itemize}
}
\end{enumerate}
\end{frame}



%\subsection{Die Vorlesung}
 \begin{frame}
 \frametitle{Übersicht über die Vorlesungen (jeweils Montag 10-12 Uhr)}%
    \begin{table}
  %  \caption{Mit zwei Ausnahmen findet die Vorlesung $14$-tägig statt}
\begin{center}
\caption{\colorbox{yellow!40}{Es gibt ein Aufgabenblatt zur Abgabe}}
  \begin{tabular}{|c|c|c|}
    \hline
    Nr. & Datum & Thema \\ \hline
    0 &   $ 16.10.2017$   &  \colorbox{blue!10}{  Der Zufallsvorgang und der Inferenzschluss}   \\ \hline
    \colorbox{yellow!40}{1} &  $23.10.2017$  &   \colorbox{blue!10}{Die parametrische  Wahrscheinlichkeitsverteilung } \\ \hline
    \colorbox{yellow!40}{2} &   $13.11.2017$   &   \colorbox{green!40}{Grundlagen des statistischen Testens}\\ \hline
    \colorbox{yellow!40}{3} &   $27.11.2017$   &   \colorbox{green!40}{Punktschätzer und Konfidenzintervalle }\\ \hline
%    \multirow{2}{*}{3} &  \multirow{2}{*}{$27.11.2017$}& \colorbox{red!20}{Grundlagen des statistischen Testens}\\
%               &               & von quantitativen Merkmalen\\ \hline
  \multirow{2}{*}{\colorbox{yellow!40}{4}} &  \multirow{2}{*}{$11.12.2017$}& \colorbox{violet!40}{Bivariate lineare Regression} \\
                   &        & \colorbox{violet!40}{und Varianzanalyse}\\ \hline  
    \colorbox{yellow!40}{5} & $08.01.2018$      &  \colorbox{violet!40}{Multiple lineare Regression}  \\ \hline   
    6 &   $22.01.2018$     &  \colorbox{violet!40}{Varianzanalyse} \\ \hline
  \end{tabular}
  \end{center}
  \label{tab:multicol}
  \end{table}
\end{frame}





\begin{frame}
 \frametitle{Übersicht über die Übungsgruppen }%
 %Die Übung findet dienstags statt, jeweils von 10-12 Uhr und 14-16 Uhr
 \begin{itemize}
 \item{Die Übung findet 14-tägig statt (1. Woche Gruppen 1 und 3, 2.Woche Gruppen 2 und 4 )}
 \item{Die Übung findet dienstags statt, jeweils von 10-12 Uhr (1 und 3) und 14-16 Uhr (2 und 4)}
 \end{itemize}
    \begin{table}
 %   \caption{Die Übung findet $14$-tägig statt}
\begin{center}
 \begin{tabular}{|c|c|c|}
    \hline
   Übung Nr.    &    Datum Gruppe   1 u. 3 &  Datum  Gruppe  2 u. 4      \\ \hline
    $1$   &   $24.10.2017$   &  07.11.2017         \\ \hline
    $2$   &   $14.11.2017$   &  21.11.2017          \\ \hline
    $3$   &   $28.11.2017$   &  05.12.2017         \\ \hline
    $4$   &   $12.12.2017$   &  19.12.2017         \\ \hline
    $5$   &   $09.01.2018$   & 16.01.2017           \\ \hline
    $6$   &   $23.01.2018$   & 30.01.2017           \\ \hline
  \end{tabular}
  \end{center}
  \label{tab:multicol}
  \end{table}
\end{frame}

\begin{frame}
 \frametitle{Übersicht über die Tutorien }%
 %Die Übung findet dienstags statt, jeweils von 10-12 Uhr und 14-16 Uhr
 \begin{itemize}
 \item{Das Tutorium findet 14-tägig statt (1. Woche Gruppen 1 und 3, 2.Woche Gruppen 2 und 4 )}
 \item{Das Tutorium findet dienstags statt, jeweils von 16-18 Uhr  und 18-20 Uhr }
 \item{Der Besuch ist freiwillig und sehr zu empfehlen, sie dürfen mehre Tutorien besuchen.}
 \end{itemize}
    \begin{table}
 %   \caption{Die Übung findet $14$-tägig statt}
\begin{center}
 \begin{tabular}{|c|c|c|}
    \hline
   Übung Nr.    &    Datum Gruppe   1 u. 3 &  Datum  Gruppe  2 u. 4      \\ \hline
    $1$   &   $24.10.2017$   &  07.11.2017         \\ \hline
    $2$   &   $14.11.2017$   &  21.11.2017          \\ \hline
    $3$   &   $28.11.2017$   &  05.12.2017         \\ \hline
    $4$   &   $12.12.2017$   &  19.12.2017         \\ \hline
    $5$   &   $09.01.2018$   & 16.01.2017           \\ \hline
    $6$   &   $23.01.2018$   & 30.01.2017           \\ \hline
  \end{tabular}
  \end{center}
  \label{tab:multicol}
  \end{table}
\end{frame}

\begin{frame}{Die Abgabe der Aufgabenblätter}
\begin{itemize}
 \item{Sie geben wieder in Gruppen (3-4 Personen) ab, in diesem Semester gibt es ein standardisiertes Deckblatt, dieses finden Sie unter \dots}\pause
  \item{Falls Sie das Aufgabenblatt aus gesundheitlichen Gründen nicht abgeben können,
  wird das auf dem Deckblatt vermerkt.}\pause
  \item{Bitte Gruppe anmelden bei \dots}
  \item{Die Lösungen werden in den Übungen zurückgegeben. Bei der Anmeldung wird festgelegt,
  in welcher Gruppe ihr Blatt zurück gegeben werden soll.}
  \item{Das letzte Aufgabenblatt ist das Blatt 5, es erscheint am Freitag den 5. Januar. 
  Die Abgabe des letzten Aufgabenblattes ist am Freitag den 19. Januar}\pause
\end{itemize}
\end{frame}

\begin{frame}
 \frametitle{Die Klausur}
 \begin{itemize}
%\item {In der Woche vom 22. Januar bis 26. Januar erfolgt die Zulassung}\pause
\item {Sie müssen bei vier von fünf Aufgabenblättern mindestens $50\%$ richtig gelöst haben, um an der Klausur teilnehmen zu dürfen.}\pause
  \item {Die Klausur findet am Montag den 12. Februar um 10-12 Uhr statt.}\pause
  \item {Die Nachholklausur findet am Montag den 12. Februar um 10-12 Uhr statt.}\pause
  \item {Das Aufgabenblatt 6 wird durch eine Probeklausur ersetzt. 
  \begin{itemize}
 % \item{So können Sich mehr auf die Art der Aufgaben einstellen. }
%  \item{Die Art der Aufgaben für die Aufgabenblätter und die Klausur orientiert sich an den Kernkompetenzen.}
  \item{Es wird dann auch vom Umfang einer erwarteten Bearbeitungszeit von 90
  Minuten entsprechen.}
  \item{Zeilenabstände werden größer sein.}
  \item{Sie brauchen keine \texttt{STATA}-Befehle zu kennen, sollen aber 
  \texttt{STATA}-Ouput interpretieren können.}
  \end{itemize}
   }
  \end{itemize}

\end{frame}
% http://minisconlatex.blogspot.de/2012/05/como-cambiar-el-color-de-una-palabra.html
\begin{frame}
 \frametitle{Mein Kontaktdaten}
 \begin{block}{Kontakt}
   \begin{enumerate}
  \item {Sprechstunde: Freitag von $10-11$ Uhr}
  \item {Mein Büro: Bachstraße 18k, 1. OG, R132}
  \item {Telefonnummer: 03641 9 45013}
  \item {E-Mail-Adresse: \url{mariana.nold@uni-jena.de}}
 \end{enumerate}
 \end{block}
   \begin{variableblock}{Material zur Vorlesung und Übung}{bg=Gray!20,fg=black}{bg=LimeGreen!50,fg=black}
  wo ?  %  \url{www.dt-workspace.de} (nur Registrierung nötig) %
    \end{variableblock}


 % www.dt-workspace.de (nur Registrierung nötig)
 %Sprechzeiten, Wo ist das Büro, E-Mail, Homepage wo das Skript zu finden ist und die Übungsblätter \dots
 %Wo befindet sich der Briefkasten für die Übungsblätter?
\end{frame}
%\section{Organisatorisches}
%\begin{frame}{Organisatorisches}
%\end{frame}
%
%\begin{frame}{Organisatorisches}
%\end{frame}
%
%\begin{frame}{Organisatorisches}
%\end{frame}

% Lage und Streuungsmaße, emp. Verteilungsfunktion
% S. 159, Mittag
\section{Rückblick: Deskriptive Statistik}



\begin{frame}
 \frametitle{Univariate deskriptive Statistik: Den Datensatz verstehen} %Wer redet, sät - und wer hört, erntet. (Sprichwort aus Argentinien)
 %\textbf{statistische Methoden sind brauchbare Werkzeuge,
 % um Information zu ordnen und zusammen zu fassen}
%  \begin{block}{Zitat von Elisabeth Noelle-Neumann, Pionierin der Demoskopie}
%   Statistik ist für mich das Informationsmittel der Mündigen. 
%   Wer mit ihr umgehen kann, kann weniger leicht manipuliert werden. \\
%   Der Satz: ``Mit Statistik kann man alles beweisen'' gilt nur für die Bequemen, die keine Lust haben, genau hinzusehen.
%  \end{block}

 %\begin{block}{Selbstbewusst durch gute Grundkenntnisse}
  \begin{itemize}
   \item {Man braucht deskriptive Methoden um den Datensatz kennen zu lernen:
   \begin{itemize}
   \item{Was bedeuten die Merkmale im Datensatz inhaltlich?}\pause
   \item{Wie wurden latente Merkmale operationalisiert?}\pause
   \item{Auf welchem Messniveau wurden sie erhoben?}\pause
   \item{Gibt es unrealistische Messwerte? (Beispiel: Person gibt an $100$ Kinder zu haben)}
   \end{itemize}
   }\pause
    \item {Wir haben im letzten Semester  Lage- und Streuungsmaße für die univariate
    Deskription kennen gelernt.}\pause
    \item{Die univariate Deskription wird mit Hilfe von Grafiken, wie dem Boxplot, der empr.
    Verteilungsfunktion, dem Balkendiagramm oder Histogramm illustriert.}
   
   % \item {Und dafür brauch man Sachverstand und kaum Mathematik.}
%    \item {Wirklich spannend ist auch, dass man die Datensätze (und ihre Geschichten) oft unterschiedlich auslegen
%    und interpretieren kann.}
  \end{itemize}
 % \end{block}
  
\end{frame}

\begin{frame}{Bivariate deskriptive Statistik: Erste Eindrücke von Zusammenhängen}
\begin{itemize}
 \item {Kreuztabellen und insbesondere die Spaltenprozentuierung sind wichtige Werkzeuge 
    der bivariaten Deskription für zwei qualitative Merkmale.}\pause
    \item{Gruppierte Balkendiagramme dienen der Veranschaulichen der  Spaltenprozentuierung.}\pause
    \item{Der Vergleich von Boxplots von verschiedenen Gruppen dient der bivariaten Deskription 
    von einem qualitativen und einem metrischen Merkmal.}\pause
    \item{Zwei metrische Merkmale können mit Hilfe der Korrelation und der Regression beschrieben werden.}
    \item{Das Streudiagramm (mit Regressionsgerade) dient der grafischen Analyse.}
\end{itemize}
\end{frame}

\begin{frame}{Die beschreibende Statistik und ihre Intention}
Die deskriptive Statistik hat die Intention einen gegebenen Datensatz zu beschreiben, dazu nutzt sie
\begin{itemize}
\item{Lagemaße wie Mittelwert, Modus, Median und andere empirische Quantile}\pause
\item{Streuungsmaße wie empirische Varianz oder Interquartilsabstand}\pause
\item{Kreuztabellen und bedingte relative Häufigkeiten, insbesondere die Spaltenprozentuierung}\pause
\item{Grafiken wie die empirische Verteilungsfunktion, Boxplots, Histogramme
oder (gruppierte) Balkendiagramme}\pause
\item{Eine \colorbox{yellow!40}{\textbf{gute}} Deskription ist eine 
\colorbox{yellow!40}{\textbf{notwendige}} Voraussetzung für die 
sinnvolle Analyse des Datensatzes.}
\end{itemize}
\end{frame}

\begin{frame}{Sie möchten Begriffe  nachschlagen
oder wiederholen?}
\begin{enumerate}
\item{(ThULB Jena) Buch:\\ \textbf{Statistik: Eine Einführung für Sozialwissenschaftler (Grundlagentexte Soziologie)},\\
 Wolfgang Ludwig-Mayerhofer , Uta Liebeskind , Ferdinand Geißler }\pause
 \item{ThULB Jena E-Book: \\
 \textbf{Statistik-
 Eine Einführung mit interaktiven Elementen}, Hans-Joachim Mittag}\\ \pause
  \item{ThULB Jena E-Book: \\
   \textbf{Statistik für Soziologen} (2. Auflage), Rainer Diaz-Bone}\pause
   \item{ThULB Jena E-Book: \\
   \textbf{Statistik- Der Weg zur Datenanalyse} (8. Auflage), Ludwig Fahrmeir und andere}
\end{enumerate}
\end{frame}

\begin{frame}{Lexikon und Online-Karteikarten}
\begin{enumerate}
\item{Online Lexikon von Prof. Ludwig-Mayerhofer: \url{http://wlm.userweb.mwn.de/Ilmes/}}\pause
\item{Karteikarten online: \url{https://www.cobocards.com/pool/de/cardset/7yua50512/online-karteikarten-33209-statistik-i/}}
\end{enumerate}
\end{frame}
\section{Statistische Inferenz}
\subsection{Der klassische Inferenzschluss}
\begin{frame}
 \frametitle{Was ist eigentlich Statistik?}
 \begin{block}{Zwei Wiener Statistiker haben sich mit dieser Frage beschäftigt}
  Während  die  meisten  Wissenschaften  eine  zumindest  \textbf{formal  klare  
Definition besitzen} und sich somit von 
anderen Wissenschaften abgrenzen können, 
gelingt dies bei der Statistik nicht so einfach. 
Ein Anhaltspunkt dafür sind schon die 
vielen sehr unterschiedlichen Definitionen 
von ``Statistik'', die in den verschiedenen 
Lehrbüchern  zu  finden  sind. 
 \end{block}
 (Quelle: Marcus Hudec, Christian Neumann, Institut für Statistik der Universität Wien)
\begin{block}{Fazit von Hudec und Neumann \dots}
  Statistik wird  als Hilfswissenschaft aufgefasst. Sie ist eine der
  Methoden, mit der die Theorie und Erfahrung (Empirie) systematisch
  reflektiert wird.
  \end{block}
\end{frame}
% Yudi Pawitan: Wie gehen wir mit Unsicherheit um?
\begin{frame}{Die schließende Statistik und ihre Bedeutung}
Welche Bedeutung bzw. Aufgabe hat die Statistik als Wissenschaft? 
\begin{block}{Antwort von Yudi Pawitan}
Uncertainty is pervasive in problems that deal with the real world, but statistics is the only branch of scinence that puts systematic effort into dealing with uncertainty.\\
Statistics is suited to problems with \textbf{inherent uncertainty due to limited inforamtion}, 
\begin{enumerate}
\item{it does not aim to remove uncertainty,}
\item{ but in many cases it merely quantifies it,}
\item{uncertainty can remain even after
analysis is finished.}
\end{enumerate}
 \end{block}
%Ich möchte Ihnen in diesem Semester einen Einblick in methodischen Ansätze (systematic efforts) geben, die die klassische Inferenz bietet. 
%Die klassische Inferenz ist
%ein Hauptzweig der induktiven Statistik.
\end{frame}



\begin{frame}
 \frametitle{Was wird als gültiger Zusammenhang anerkannt?}
 
  
  \begin{block}{\dots im Laufe der Zeit immer wichtiger wird}
   Die Bedeutung der Statistik ist zur Zeit einem großen Wachstum
   unterworfen.
   \end{block}
   \begin{variableblock}{Ein Grund für die zunehmende Bedeutung}{bg=GreenYellow!70,fg=black}{bg=YellowGreen!70,fg=black}
     Statistische Methoden sind oft entscheidend dafür, 
   welche Zusammenhänge in der Gesellschaft als gültig anerkannt werden.
    \end{variableblock}
   Beispiel:
   \begin{enumerate}
   \item{Pisa-Studie: Sollte man Geld investieren, um die Jungs bei Lesen lernen zu unterstützen?}
   \item{Unfallstatistik: Sollte man Geld investieren, um für mehr Sicherheit auf deutschen Autobahnen
   zu sorgen?}
   \end{enumerate}

\end{frame}
%\begin{frame}{Modelle in deskriptiver und induktiver Statistik} % spielen sehr zentrale Rolle in der induktiven Statistik
%\end{frame}

\begin{frame}{ Beispiel: PISA-Studie }%und verschieden Inferenzkonzepte
%Uncertainty is pervasive in problems that deal with the real world:\\ 
Warum sind viele wissenschaftliche Fragestellungen von Unsicherheit durchdrungen?
\begin{itemize}
%\item{Wenn wir Aussagen über die reale Welt machen, dann haben wir nur einen Teil der möglichen Information. Wir haben in der Regel eine Stichprobe
%und möchten eine Aussage über die Grundgesamtheit machen.}
\item{Wir wollen basierend auf den PISA-Daten  eine Antwort auf die Frage finden, ob die Leseleistung von 15-jährigen  Mädchen
in Deutschland besser ist als die von 15-jährigen  Jungen.}\pause
\item{Unsere Stichprobe enthält nicht die Information über die Leseleistung aller 15-Jährigen in Deutschland.\\
(inherent uncertainty due to limited inforamtion) }\pause
\item{Wenn wir eine Aussage über die Grundgesamtheit machen,
kann diese Aussage falsch sein. Die Wissenschaftlichkeit der Aussage entsteht dadurch, dass wir ihre Unsicherheit quantifizieren. }
\end{itemize}
\end{frame}
% http://www.spiegel.de/auto/aktuell/unfallstatistik-2015-erneut-mehr-verkehrstote-in-deutschland-a-1079184.html
% http://www.spiegel.de/auto/aktuell/zahl-der-verkehrstoten-sinkt-mehr-unfalltote-auf-autobahnen-a-955494.html
\begin{frame}{ Beispiel: Unfalltod  auf der Autobahn}
%Warum sind viele wissenschaftliche Fragestellungen von Unsicherheit durchdrungen?
\begin{itemize}
\item{Wir wollen eine Aussage darüber machen, ob die Wahrscheinlichkeit für den Unfalltod auf deutschen Autobahnen gestiegen ist.}\pause
\item{ Die Zahl der Unfalltoten ist von \colorbox{yellow!40}{$358$} (2012) auf \colorbox{yellow!40}{$387$}
 (2013) gestiegen. 
Ist das dann eine ``wirkliche'' Veränderung? }\pause
\item{Die Statistik fragt: Hat sich etwas an dem stochastischen Mechanismus geändert der diese Unfalltoden hervorbringt?}\pause
\item{Um wie viel muss die Zahl steigen bzw. sinken, damit man davon ausgeht, dass sich dieser Mechanismus verändert hat?}\pause
\item{Diese Frage kann nicht willkürlich beantwortet werden. Es muss klar sein, wann man von einer signifikanten Veränderung spricht und was damit gemeint ist.}
%\item{}
\end{itemize}
\end{frame}
%poisson.test(c(358, 387), c(1, 1), alternative = c("less"))
\begin{frame}{Inferenzkonzepte}
\begin{itemize}
\item{Durch welche Legitimation ist es erlaubt davon zu sprechen, dass Mädchen besser lesen als Jungen oder dass die Anzahl der Unfalltoten signifikant gestiegen ist?}\pause
\item{Wenn ich die entsprechenden statistischen Tests rechne, dann ist der Unterschied in der Leseleistung signifikant, der Unterschied in der Zahl der Unfalltoden nicht.}\pause
\item{Was bedeuten solche Aussagen bzw. Behauptungen und wodurch werden sie legitimiert?}\pause
\item{Unter einem Inferenzkonzept versteht man ein statistisches Konzept, das die Schlussfolgerung von Beobachtungen (Daten) auf inhaltliche Aussagen rechtfertigt.}
\end{itemize}
\end{frame}



\begin{frame}{Klassische Inferenz}
\begin{itemize}
\item{Wie werden innerhalb der Statistik Schlüsse von Beobachtungen auf Hypothesen gerechtfertigt?}\pause
\item{Angesichts der Vielfalt wissenschaftstheoretischer und philosophischer Ansätze zur Erklärung des empirischen Forschungsansatzes
ist es nicht verwunderlich, dass es auf diese Frage innerhalb der Statistik mehre Antworten gibt und 
\textbf{nicht eine einzige, in sich geschlossene statistische Inferenztheorie
existiert. }}\pause% Rüger I S. 116,117
\item{Die drei bekanntesten Inferenztheorien sind die Likelihood-Inferenz, die Bayes-Inferenz und die Klassische Inferenz.}
\item{Wir beschäftigen uns in diesem Semester mit Ansätzen aus der Klassischen Inferenz.}
\end{itemize}
\end{frame}
% Mittag S. 35
\begin{frame}{Der Inferenzschluss der Klassischen Inferenz}
\begin{itemize}
\item{Um ein valides  Abbild der Grundgesamtheit zu bekommen,
zieht man eine Zufallsstichprobe.}\pause
\item{Nur bei Realisierung einer Zufallsauswahl kann von einer Stichprobe
\textbf{mit kontrollierter Irrtumswahrscheinlichkeit} auf die zugrundeliegende
Grundgesamtheit zurückgeschlossen werden. }\pause
\item{Dieser auch \textbf{Inferenzschluss} bezeichnete Rückschluss
von Eigenschaften der Stichprobe auf Eigenschaften der Grundgesamtheit ist
Gegenstand der induktiven Statistik.}\pause
\item{Ein Inferenzschluss ist stets mit Unsicherheit behaftet. Dies ergibt sich
zwingend daraus, dass nur eine Teilinformation zu Verfügung steht.}\pause
\end{itemize}
Leseempfehlung:
(\url{http://www.fernuni-hagen.de/ksw/neuestatistik/content/MOD_27531/html/comp_27634.html})
\end{frame}

\subsection{Die Intention der empirischen Forschung}
\begin{frame}
 \frametitle{Empirische Forschung: Muster in Daten erkennen und bewerten}
 \begin{enumerate}
  \item {Die Statistik stellt Methoden zur Verfügung um Information aus Beobachtungen
  gut zusammen zu fassen und diese Information zielgerichtet zu verarbeiten.}
 % \item {Information aus Beobachtungen nennt man \\ \textbf{empirische Information.}}
  \item {Die Statistik gibt uns Werkzeuge um empirische Information übersichtlich darzustellen.
  (Deskriptive Statistik)}
  \item {Beruhend auf einer übersichtlichen Darstellung können wir z. B. Vermutungen über Zusammenhänge
  ableiten. (Explorative Statistik)}
  \item {Die Statistik erlaubt u. U. von der beobachteten empirischen Information  auf eine über die Stichprobe hinaus gültige Aussagen zu schließen. (Induktive Statistik)}
 \end{enumerate}
\end{frame}

\begin{frame}{Empirische Forschung ist vergleichbar damit ein Mosaik zu restaurieren}
     \begin{figure}[ht]
 	\centering
 	      \includegraphics[width=0.65\textwidth]{incomplete.jpg}
 	\end{figure}
\end{frame}

\begin{frame}{Restauration: Was es zu beachten gilt}
\begin{enumerate}
\item{Transparenz ist wichtig. Man muss sagen, welche Teile man zu Beginn hatte und durch welche 
Annahmen und Methoden man zu dem Gesamtbild kommt.}\pause
\item{Das Bild das, nach der Restauration entsteht, kann (in Teilen) fehlerhaft sein. Es ist mit Unsicherheit behaftet.}\pause
\item{Je weniger Teile man zu Beginn hat bzw. je komplizierter das Bild ist, desto größer ist die Unsicherheit über das Ergebnis.}\pause
\item{Es gibt  viele Möglichkeiten zu dem Ergebnis zu kommen, es gibt nicht den einen richtigen Weg,
auch wenn es nur ein richtiges Bild gibt.}\pause
%\item{Konstruktiv-kritisches Nachfragen daher ist sinnvoll. Man braucht den Blick von unterschiedlichen Personen (aus verschiedenen Disziplinen)
%um die Güte des Bildes beurteilen zu können.}
\end{enumerate}
\end{frame}

\begin{frame}{Gibt es überhaupt ein Muster?}
     \begin{figure}[ht]
 	\centering
 	      \includegraphics[width=0.6\textwidth]{pattern.jpg}
 	\end{figure}
\end{frame}

\begin{frame}{Können wir aus den gegebenen Informationen etwas schließen?}
     \begin{figure}[ht]
 	\centering
 	      \includegraphics[width=0.65\textwidth]{incomplete2.jpg}
 	\end{figure}
\end{frame}

\begin{frame}{Wie viel darf hier fehlen, damit man eine Chance hat das Muster zu finden?}
     \begin{figure}[ht]
 	\centering
 	      \includegraphics[width=0.65\textwidth]{mosaic-pattern.jpg}
 	\end{figure}
\end{frame}

\begin{frame}{Welche Teile dürfen fehlen, damit man eine Chance hat das Muster zu finden?}
     \begin{figure}[ht]
 	\centering
 	      \includegraphics[angle=90,width=0.6\textwidth]{patchwork.jpg}
 	\end{figure}
\end{frame}



\begin{frame}{Werkzeugkiste für die Restauration}
Wir wollen einen Ausschnitt der sozialen Wirklichkeit erforschen und erheben Daten,
um aus dieser  ``limited information'' eine Aussage über die dahinter stehenden 
Mechanismen machen.\\
Was gehört in die Werkzeugkiste?\pause
\begin{enumerate}
\item{Kreativität und Fingerspitzengefühl im Umgang mit den Daten}\pause
\item{Sachwissen über den Hintergrund der Daten (inhaltliches Wissen)}\pause
\item{Grundlagen der Methoden der empirischen Sozialforschung}\pause
\item{Grundlagen der deskriptiven Statistik}\pause
\item{Grundlagen der Inferenzstatistik, insbesondere der Test- und Schätztheorie}\pause
\item{Die Fähigkeit Statistik-Software zu nutzen}
\end{enumerate}

\end{frame}

\section{Zufallsvorgang}% Fahrmeier-Tutz S. 174, Yudi Pawitan S.4 manchmal ist es auch einfach zu komplex

\begin{frame}{Beispiel: Befragung nach Höhe des Taschengeldes}
\begin{itemize}
\item{Fiktiv: In einem Artikel einer regionalen Zeitung aus dem Jahr 2015 steht, dass Grundschulkinder in Jena im Durchschnitt
elf Euro Taschengeld im Monat gekommen.}\pause
\item{Wir möchten empirische Information zu diesem Thema sammeln und befragen 
$n=100$ Grundschulkinder in Jena nach ihrem monatlichen Taschengeld (gerundet auf ganze Euro).}\pause
\item{Zwei Eigenschaften kennzeichnen den Vorgang:}\pause
\begin{enumerate}
\item{Man kennt im Vorfeld bereits die möglichen Ausgänge, wobei unbekannt ist, welcher Eintritt.}
\item{Es hängt ``vom Zufall'' ab, welchen Ausgang man beobachtet.}
\end{enumerate}
\end{itemize}
\end{frame}

\begin{frame}{Zufallsvorgang}

Gerade diese beiden Eigenschaften kennzeichnen den Zufallsvorgang. Das Zufallsexperiment ist ein Spezialfall
des Zufallsvorgangs.
\begin{variableblock}{Definition: Zufallsvorgang }{bg=Orchid!30,fg=black}{bg=Plum!30,fg=black}
    	Ein Zufallsvorgang führt zu einem von mehreren, sich gegenseitig ausschließenden Ergebnissen. Es ist vor der Durchführung ungewiss, welches Ergebnis
    	tatsächlich eintreten wird.
\end{variableblock}\pause
\begin{itemize}
\item{Die Ziehung einer Stichprobe ist ein Beispiel für einen Zufallsvorgang. }\pause
\begin{itemize}
\item{ Befragung von Grundschulkindern nach der Höhe ihres Taschengeldes}\pause
\item{Ergebnisse des Kompetenztest Lesen in der PISA-Studie}
\end{itemize}
\item{Viele Prozesse die wir beobachten können sind Zufallsvorgänge. }
\begin{itemize}
\item{Beispiel: Wie viel Personen nehmen heute an der Vorlesung teil? }\pause
\item{Wie viele Unfalltode gab es im Jahr 2016?}
\end{itemize}
\end{itemize}
\end{frame}



% Durchschnittswert abziehen, dass Verteilung um Null schwankt. Der ist uns bekannt aus der Zeitung

% Zufallsvariable: Beispiel Taschengeld, Beispiel Normalverteilung
%\section{Zufallsvorgang am Beispiel}




\begin{frame}{Zufallsvariable}
Ein ganz zentraler Begriff in diesem Semester ist der der Zufallsvariable:
\begin{variableblock}{Definition: Zufallsvariable }{bg=Orchid!30,fg=black}{bg=Plum!30,fg=black}
    	Eine Variable oder ein Merkmal $X,$ dessen Werte oder Ausprägungen die Ergebnisse eines 
    	Zufallsvorgangs sind, heißt Zufallsvariable $X.$ Die Zahl $x,$ die $X$ bei der Durchführung
    	eines Zufallsvorgangs annimmt, heißt Realisierung oder Wert von $X$
\end{variableblock}
Für unser Beispiel definiere ich die Zufallsvariable  $X:$
\begin{equation}
X := \text{Höhe des Taschengeld von Grundschulkind aus Jena in  Euro}
\end{equation}

\end{frame}

\begin{frame}{Die Notation der Zufallsvariable und ihrer Realisation}
\begin{itemize}
\item{Großbuchstaben bezeichnen  den Zufallsvorgang selbst. Sie also eine Abkürzung
für den Zufallsvorgang. }
\item{Die entsprechenden Kleinbuchstaben werden mit einem Index versehen und sind die Werte,
die konkret beobachtet wurden.}
\item{So bedeutet z. B. $x_{1}=14,$ dass das erste Kind in der Befragung $14$ Euro Taschengeld
im Monat bekommt.}
\item{Entsprechend bedeutet $x_{2}=9,$ dass das zweite Kind $9$ Euro Taschengeld bekommt.}
\end{itemize}
\end{frame}



\begin{frame}{Die Intention der deskriptive Modellierung}

\begin{itemize}
\item{Man beschreibt die beobachteten Daten mit Hilfe einer Modellverteilung.}\pause
\item{Ziel ist es, die in den den Daten bestehen Sachverhalte zu kommunizieren.}\pause
\item{Ziel ist es \textbf{nicht,} eine Aussage zu machen, die über das Beschreiben der Daten
hinausgeht.}\pause
\item{Insbesondere bezieht sich ein deskriptives Modell nur auf die vorliegende Stichprobe.}
%\item{Wie hatten}
\end{itemize}

\end{frame}

\begin{frame}{Modellebene und Datenebene}

\begin{itemize}
\item{Die Normalverteilung ist ein Modell, dass sich für symmetrische
Verteilungen eignet.}\pause
\item{Die Parameter $\mu$ und $\sigma$ sind Platzhalter
für den Ort der Symmetrieachse und die Breite der Dichte.}\pause
\item{Wenn man diese Parameter aus Daten schätzt,
werden die Schätzer üblicherweise mit $\hat{\mu}$
und $\hat{\sigma}$ bezeichnet.}\pause
\item{Durch die  Spezifikation der Parameter mit Hilfe der
Daten wendet man das Modell auf der Datenebene an.}\pause
\item{Datenebene bedeutet: Man nimmt Bezug zu den vorliegenden Daten}
\item{Modellebene bedeutet: Man spricht über Modellannahmen}
%\item{Bei einer deskriptiven Modellierung dient es 
%ausschließlich der Beschreibung
%der vorliegenden Daten.}
\end{itemize}
\end{frame}

\begin{frame}{Ergebnis der fiktiven Erhebung in Zahlen}
Wir haben die fiktive Erhebung zum Thema Taschengeld abgeschlossen und wollen herausfinden,
ob sich die Normalverteilung eignet um die Daten zu beschreiben.
\begin{itemize}
\item{Ist die beobachtetet Verteilung der Daten symmetrisch?}
\begin{itemize}
\item{Mittelwert: $\bar{x}=12.33,$ Median (= $50\%$ Quantil): $\tilde{x}=12,34$}\pause
\item{1. Quartil (= $25\%$ Quantil): $x_{0.25}=10.52$}\pause
\item{3. Quartil (= $75\%$ Quantil): $x_{0.75}=14.07$}\pause
\item{Minimum: 5.36 und Maximum: 19.2}\pause
\end{itemize}
\item{Sprechen die Lagemaße für eine symmetrische Verteilung?}\pause
\item{Grafische Verfahren werden genutzt um einen Eindruck von den Daten zu bekommen. Wir setzen $\mu \equiv \hat{\mu}=\bar{x}=12.33$ und $\sigma^{2} \equiv \hat{\sigma}^{2}=7.26.$}\pause
\item{Das Zeichen $\equiv$ bedeutet, dass wir für den Modellparameter einen bestimmten Wert einsetzen.}
\end{itemize}
%Lage- und Streuungsmaße (auch Quantile)
% Begriff Zufallsvariable Mittag S. 156, FT S. 226
%VORSICHT: mean und sd nicht echt sondern geschätzt!!!
\end{frame}
%% Man fragt in einer Schule Schülerinnen und Schüler die Vorbei kommen, danach, wie viel Taschengeld sie bekommen
%
%%\begin{frame}{Was möchte eine statistisches Modell}
%%\end{frame}
%
%\begin{frame}{Verteilung einer Zufallsvariable} % Mittag S. 156
%
%\end{frame}



\begin{frame}{Grafische Darstellung: Empirische Verteilungsfunktion und Boxplot}
        \begin{figure}[ht]
 	\centering
 	      \includegraphics[width=0.55\textwidth]{taschengeld_box.pdf}
 	\end{figure}
\end{frame}

\begin{frame}{Grafische Darstellung: Empirische Verteilungsfunktion und theoretische Verteilungsfunktion}
        \begin{figure}[ht]
 	\centering
 	      \includegraphics[width=0.55\textwidth]{taschengeld_theo.pdf}
 	\end{figure}
\end{frame}

\begin{frame}{Grafische Darstellung: Histogramm und deskriptive Normalverteilung}
        \begin{figure}[ht]
 	\centering
 	      \includegraphics[width=0.65\textwidth]{taschengeld_dens.pdf}
 	\end{figure}
\end{frame}


% siehe: LösungÜbung3Neu.pdf
\begin{frame}{Was haben wir bisher gemacht?}
\begin{itemize}
\item{Wir haben $n=100$ Realisationen der Zufallsvariable Taschengeld beobachtet
und beschreiben.}\pause
\item{Bisher haben wir nur Methoden der deskriptiven Statistik genutzt. Wir machen also
noch keine Aussage die über die Stichprobe hinausgeht.}\pause
\item{Wir nutzen die Normalverteilung als Modell um unsere Daten zu beschreiben
und gewinnen den Eindruck, dass diese Modell gut zu den Daten passt}\pause
\item{Wie können Sie die 68-95-99.7-Regel nutzen um die Daten zu beschreiben?}
\end{itemize}
\end{frame}

\begin{frame}{Darf ich aus der Stichprobe eine allgemeinere Aussage machen?}
\begin{itemize}
\item{Können wir mit Bezug auf unsere Stichprobe behaupten, dass sich das Taschengeld 
der Grundschulkinder signifikant erhöht hat?}\pause
\item{Es waren im Durchschnitt elf Euro im Jahr 2015.}\pause
\item{Eine Zufallsstichprobe erlaubt genau das zu tun. }\pause
\item{Bei der Art und Weise zu schließen hat man verschiedene 
Inferenzkonzepte zu Auswahl. Oft führen sie zu ähnlichen Schlüssen.}
\end{itemize}
\end{frame}

\begin{frame}{Die Zufallsstichprobe}
\begin{block}{Bedingungen für eine Zufallsauswahl (aus Methoden I)}
\begin{enumerate}
\item{Regelgeleitete Auswahl und}
\item{für jedes Element der Grundgesamtheit ist die Wahrscheinlichkeit in die Stichprobe
aufgenommen zu werden bekannt und}
\item{größer Null}
\end{enumerate}
\end{block}
\begin{variableblock}{Definition: Einfache Zufallsstichprobe}{bg=Orchid!30,fg=black}{bg=Plum!30,fg=black}
Eine einfache Zufallsstichprobe liegt dann vor, wenn jede der möglichen auf diese Weise gezogenen Stichproben vom Umfang n aus einer Grundgesamtheit vom Umfang $N$ die gleiche Chance hat aufzutreten. 
\end{variableblock}

\end{frame}


\section{Plan für die nächsten Wochen}
\begin{frame}{Was wir in den nächsten drei Wochen machen (bis 10.11.2017)}
%%Wird in der Übung wiederholt
\begin{itemize}
%\item{ha}
\item{Wir wiederholen den zentralen Zusammenhang: Theoretische Dichte und Verteilungsfunktion}
\item{Wir nutzen \texttt{STATA} um den fiktiven Taschengelddatensatz zu beschreiben}
\item{Es wird auch das Geschlecht erhoben, so dass wir die Verteilung des
Taschengelds in den beiden Gruppen vergleichen können.}
\item{In der nächsten Vorlesung lernen Sie das Modell der Binomialverteilung kennen.}
\item{In der Übung werden wir auch dieses Modell mit \texttt{STATA} erkunden.}
\item{Inferenzschlüsse werden noch nicht gezogen, aber wir sprechen darüber,
was es bedeutet das zu tun und unter welchen Voraussetzungen wir es tun dürfen.

}
%%}
\end{itemize}
\end{frame}
% Daten und Modellebene: Unterscheidung finde ich nicht mehr sinnvoll?
% Beispiel: Aufgabe: Ein Punkt auf Bezug zu diesem Wert.
% Umformulieren in: Daten mit Modell in Verbindung bringen: ja oder nein
% Man kann Daten deskriptiv mit Modell in Verbindung bringen

%Wir haben aufgehört mit dem Gruppenunterschied (Jungen-Mädchen: Lesen sie besser)
%Wir fangen wieder an mit Taschengeld (aktueller Bezug: http://www.spiegel.de/lebenundlernen/schule/jungen-bekommen-mehr-taschengeld-als-maedchen-a-1161915.html)
% http://www.spiegel.de/wissenschaft/mensch/taschengeld-gibt-es-wirklich-einen-gender-pay-gap-a-1100475.html
\begin{frame}{Bekommen Jungen mehr Taschengeld als Mädchen?}
Bitte lesen sie die beiden Artikel 
\begin{enumerate}
\item{ \colorbox{yellow!20}{Statistik und Wahrheit - Hauptsache spektakulär} \footnotesize{ \url{http://www.spiegel.de/wissenschaft/mensch/taschengeld-gibt-es-wirklich-einen-gender-pay-gap-a-1100475.html} } 
Erscheinungsdatum: 27.7.2016}
\item{ \colorbox{yellow!20}{ Umfrage: Jungen bekommen mehr Taschengeld als Mädchen} \footnotesize{ \url{http://www.spiegel.de/lebenundlernen/schule/jungen-bekommen-mehr-taschengeld-als-maedchen-a-1161915.html} }
Erscheinungsdatum:  8.8.2017}
\end{enumerate}
\colorbox{green!20}{Machen Sie sich ein paar Stichpunkte zu dem Ergebnis, }
\colorbox{green!20}{dass der zweite Artikel Ihnen Nahe legt und kommentieren Sie es kritisch.}
\end{frame}
\end{document}

%\begin{frame}{Zufallsexperiment}
%\begin{variableblock}{Definition: Zufallsexperiment}{bg=Orchid!30,fg=black}{bg=Plum!30,fg=black}
%    	Bei Beobachtungsstudien, Befragungen oder allgemeinen Stichprobenerhebungen sind im Gegensatz zu Experimenten die
%    	Rahmenbedingungen i. a. nicht kontrollierbar bzw. bekannt. Man spricht von einem Zufallsexperiment, wenn ein Zufallsvorgang 
%    	unter kontrollierten Bedingungen abläuft und somit unter gleichen Bedingungen wiederholbar ist.
%\end{variableblock}
%Unter Rahmenbedingungen bzw. Bedingungen sind alle Umstände zu verstehen, die einen Einfluss auf den Vorgang haben.
%Das Würfeln ist eine Beispiel für ein Zufallsexperiment.
%\end{frame}
%
%\begin{frame}{Chaotisches System}
%\begin{variableblock}{Definition: Chaotisches System}{bg=Orchid!30,fg=black}{bg=Plum!30,fg=black}
%Oft beschreiben wie unerklärte Phänomen mit Hilfe von probabilistischen Gesetzen,
%obwohl man diese Phänomene rein deterministisch beschreiben könnte. Es ist eine Möglichkeit mit Prozessen
%umzugehen, die zu komplex sind, um sie einfach vorherzusehen. Es ist dann für uns ungewiss, welches Ereignis
%tatsächlich eintreten wird, auch wenn man es theoretisch vorhersagen könnte.
%%Often, we impose a probabilistic structure on an unexplained
%%phenomenon although the structure of the phenomenon is purely
%%deterministic but the underlying mechanism is too complex and
%%cannot be recovered from the data, e.g. chaotic systems.
%\end{variableblock}
%\begin{itemize}
%\item{Wir unterscheiden nicht, ob ein Prozess eigentlich deterministisch ist und einfach zu komplex um den Ausgang absehen zu können oder
%ob es ein wirklicher stochastischer Prozess ist.}
%\item{Pendelbewegungen als auch das Würfeln können als chaotische Systeme verstanden werden.}
%\end{itemize}
%\end{frame}
%
%\begin{frame}{Zufallsvorgang}
%Der Begriff Zufallsvorgang ist daher ein Überbegriff für Zufallsexperimente und chaotische Systeme.
%Charakteristisch für den Zufallsvorgang sind zwei Eigenschaften
%\begin{enumerate}
%\item{Man kennt im Vorfeld bereits die möglichen Ausgänge, wobei unbekannt ist, welcher Eintritt.}
%\item{Es hängt ``vom Zufall'' ab, welchen Ausgang man beobachtet.}
%\end{enumerate}
%\begin{block}{Befragung nach Taschengeld als Zufallsvorgang}
%Im Folgenden wollen wir die Normalverteilung nutzen als probabilistisches Gesetzt um die fiktive Befragung
%von Kindern in einer Schule nach der Höhe ihres Taschengeldes als Zufallsvorgang zu beschreiben.
%\end{block}
%\end{frame}