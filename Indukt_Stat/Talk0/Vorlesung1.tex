% multi: https://texblog.org/2012/12/21/multi-column-and-multi-row-cells-in-latex-tables/

% Mit pdflatex mindestens 2mal uebersetzen und Ergebnis mit einem pdf-Viewer betrachten
%\documentclass{beamer}
% https://en.wikibooks.org/wiki/LaTeX/Colors
\documentclass[usenames,dvipsnames,handout]{beamer}
%\usepackage[latin1]{inputenc}
%\usepackage[ngerman]{babel}
\usepackage[utf8]{inputenc}
\usepackage[ngerman]{babel} 
\usepackage{color}
\usepackage{multirow,array}
%\usepackage{multirow}
\usepackage{hyperref}
\usepackage{tikz}
\usetikzlibrary{shapes.geometric, arrows}
\usetikzlibrary{fit,arrows,calc,positioning}
% http://tex.stackexchange.com/questions/33231/how-to-change-the-color-of-a-block-within-a-custom-beamer-sty-theme-file
\usepackage{color}
\definecolor{mygreen}{cmyk}{0.82,0.11,1,0.25}
\usetheme[secheader]{Boadilla}
\newenvironment{variableblock}[3]{%
  \setbeamercolor{block body}{#2}
  \setbeamercolor{block title}{#3}
  \begin{block}{#1}}{\end{block}}


\begin{document}
\author[Dr. Mariana Nold]{Dr. Mariana Nold}
% \begin{center}
\institute[Institut für Soziologie]{ Institut für Soziologie, Fakultät für Sozial- und Verhaltenswissenschaften, Lehrstuhl für
 empirische Sozialforschung und Sozialstrukturanalyse}
% \end{center}
 \date{}
\title [Deskriptive lineare Regressionsmodellierung und Streuungszerlegung]{Deskriptive lineare Regressionsmodellierung und  Streuungszerlegung}
\date{19. Juni 2017}
\begin{frame}
\maketitle

  \begin{figure}[ht]
 	\centering
 	      \includegraphics[width=0.15\textwidth]{index.jpeg}
 	\end{figure}
\end{frame}

\begin{frame}
  \frametitle{Übersicht}
  \tableofcontents
\end{frame}

\section{Ziele der heutigen Veranstaltung}
\begin{frame}{Ziel der heutigen Veranstaltung \dots}
ist es die folgenden Fragen beantworten zu können:
\begin{block}{Zielfragen für heute}
\begin{enumerate}
\item{Wie nutzt man die Normalverteilung als deskriptives Modell?}
\item{Wie nutzt man das Konzept des linearen Zusammenhangs mit Fehlerterm in der
deskriptiven Modellierung?}
%\item{Was bedeutet es mit Hilfe einer linearen Funktion Variabilität zu erklären?}
%\item{Was bedeutet die Methode der kleinsten Quadrate?}
\item{Wie funktioniert die Streuungszerlegung?}
\item{Wie ist das Bestimmtheitsmaß definiert?}
%\item{Was ist eine Drittvariable?}
%\item{Welche Idee der Drittvaribalenkontrolle lässt sich in die 
%deskriptive lineare Regressionsmodellierung einbetten?}
\item{Wie nutzt man die Streuungszerlegung im Fall eines metrischen Merkmals,
dass in zwei Gruppen beobachtet wird?}
\end{enumerate}
\end{block}
\end{frame}

\section{Deskriptive Modellierung}

\subsection{Normalverteilung als deskriptives Modell}
\begin{frame}{Welche Rolle hat die Normalverteilung,
wenn man sie als deskriptives Modell verwendet?}
\begin{itemize}
\item{Die Normalverteilung ist durch zwei Parameter charakterisiert:}\pause
\begin{itemize}
\item{Der Erwartungswert $\mu$ legt fest, wo die Symmetrieachse verläuft.}\pause
\item{Die Standardabweichung $\sigma$ legt die Variabilität des Merkmals fest.}\pause
\end{itemize}
\item{Die Normalverteilung ist ein attraktives Modell um symmetrische Verteilungen
zu beschreiben.}\pause
\item{Durch Festlegung einer Normalverteilung
als \underline{deskriptive Modellverteilung} beschreibt man das Merkmal d. h. man macht nur darüber eine Aussage,
wie das Merkmal in der vorliegenden Stichprobe verteilt ist.}
\end{itemize}
\end{frame}





\begin{frame}{Die Normalverteilung als deskriptive Modellverteilung}
\begin{itemize}
%\item{Wenn man einen Schätzer für $\mu$ und $\sigma$ so findet,
%dass die resultierende Dichte das Histogramm der Daten gut approximiert 
%hat man eine gute Modellverteilung gefunden.}\pause
\item{Es gibt etablierte Vorgehensweisen um $\mu$
und $\sigma$ so festzulegen, dass die vorliegenden Daten gut approximiert 
werden.}\pause
\item{Auch eine rein an der Grafik
orientierte Vorgehensweise ist erlaubt, um die Daten zu beschreiben.}\pause
\item{Mit der $68-95-99.7$-Regel kann man nach Schätzung der Parameter $\mu$
und $\sigma$  z. B. Intervalle benennen, die 
einen entsprechenden Anteil der Daten enthalten.}\pause
\item{Viele Wissenschaftler und Wissenschaftlerinnen kenne die Normalverteilung gut.
Wenn man ihnen sagt, welche Normalverteilung (durch Festlegung
der Parameter) die Daten approximiert, können
sie sich vorstellen, wie die Daten verteilt sind.}
\end{itemize}
\end{frame}

\begin{frame}{Die Parameter durch eine visuelle Vorgehensweise bestimmen}
\begin{itemize}
\item{Ein grafische Anpassung kann sich entweder
an der empirischen Verteilungsfunktion oder am
Histogramm orientieren.}\pause
\item{Man kann verschieden Möglichkeiten für
$\mu$ und $\sigma$ einsetzen und anhand der Grafik
darüber entscheiden, ob die Daten gut approximiert werden.}\pause
\item{Eine Dichte bzw. Verteilungsfunktion, die die 
Daten gut approximiert, eignet sich zu Beschreibung der Daten.}
\end{itemize}
\end{frame}

\begin{frame}{Die Intention der deskriptive Modellierung}

\begin{itemize}
\item{Man beschreibt die beobachteten Daten mit Hilfe einer Modellverteilung.}\pause
\item{Ziel ist es, die in den den Daten bestehen Sachverhalte zu kommunizieren.}\pause
\item{Ziel ist es nicht, eine Aussage zu machen, die über das Beschreiben der Daten
hinausgeht.}\pause
\item{Insbesondere bezieht sich ein deskriptives Modell nur auf die vorliegende Stichprobe.}
\end{itemize}

\end{frame}

\begin{frame}{Modellebene und Datenebene}

\begin{itemize}
\item{Die Normalverteilung ist ein Modell, dass sich für symmetrische
Verteilungen eignet.}\pause
\item{Die Parameter $\mu$ und $\sigma$ sind Platzhalter
für den Ort der Symmetrieachse und die Breite der Dichte.}\pause
\item{Wenn man diese Parameter aus Daten schätzt,
werden die Schätzer üblicherweise mit $\hat{\mu}$
und $\hat{\sigma}$ bezeichnet.}\pause
\item{Durch die  Spezifikation der Parameter mit Hilfe der
Daten wendet man das Modell auf der Datenebene an.}\pause
%\item{Bei einer deskriptiven Modellierung dient es 
%ausschließlich der Beschreibung
%der vorliegenden Daten.}
\end{itemize}
\end{frame}

\begin{frame}{Beispiel aus der Übung}


\begin{block}{ Die Aussage ist:}
Die empirische Verteilungsfunktion der Lese-Punkte der deutschen Schülerinnen und Schüler wird durch eine Normalverteilung mit
 $\hat{\mu}=507.465$ und $\hat{\sigma}=91.263$ Punkte gut beschrieben.
\end{block}\pause
Effekt: 
\begin{itemize}
\item{Andere Wissenschaftler und Wissenschaftlerinnen haben eine konkrete Vorstellung
über die Verteilung in der vorliegenden Stichprobe z. B. durch die $68-95-99.7$-Regel.}\pause
\item{Eine Aussage, die sich auf alle deutschen Schülerinnen und Schüler bezieht
wird bei einer deskriptiven Modellierung nicht gemacht.}
\end{itemize}
\end{frame}
\subsection{Die lineare Regression als deskriptives Modell}

\begin{frame}{Aufgabe 16: Die durch die Ausgleichsgerade erklärte Streuung}
\begin{itemize}
\item{In dieser Übungsaufgabe haben wir darüber gesprochen, dass man eine Gerade durch eine
Punktwolke legt um den Zusammenhang zwischen $X$ (Mathe-Punkte)
und $Y$ (Lese-Punkte) zu beschreiben.}\pause
\item{Wir hatten die Summe $\sum_{i=1}^{n}|y_{i}-\hat{y}_{i}|$ als Maß
kennengelernt, dass desto höhere Werte annimmt, desto mehr die individuellen
Beobachtungen von der Gerade abweichen.}\pause
\item{Dabei bezeichnet $\hat{y}_{i}$ den Wert, der beruhend auf der Geraden
berechnet wird. Es ist also der Wert, 
den man findet, wenn man auf der Abszisse an der Stelle
$x_{i}$ eine vertikale Linie nach oben zieht und die Stelle markiert,
an der sie die Ausgleichsgerade schneidet.}
\end{itemize}
\end{frame}

\begin{frame}{Programme heute}
\begin{block}{Die Ausgleichsgerade}
\begin{equation}
\label{eq:1}
\hat{y}_{i} = \hat{\alpha} + \hat{\beta} \cdot x_{i} 
\end{equation}
\end{block}
\begin{itemize}
\item{Bisher war die Formel für die Berechnung der Ausgleichsgerade einfach gegeben.}
\item{Heute möchte ich euch zeigen, wie man auf diese Formel kommt.}
\item{Der Baustein $|y_{i}-\hat{y}_{i}|$ spielt in diesem Kontext eine wesentliche Rolle.}
\item{Es wird sich zeigen, dass  \eqref{eq:1} die beste Gerade ist,
die man durch die Punktwolke legen kann, in dem Sinne, dass 
die Summe $\sum_{i=1}^{n}(y_{i}-\hat{y}_{i})^{2}$
minimiert wird. }
\end{itemize}

\end{frame}

\begin{frame}{Zur Erinnerung:\\ Die Aufgabenstellung der Übungsaufgabe 16}
In der Vorlesung hatten wir über den (vermuteten) linearen Zusammenhang der Mathe- und Lese-Punkte gesprochen. Die Grafik \ref{fig2} zeigt für die ersten $300$
Schülerinnen und Schüler den Zusammenhang in einem Streudiagramm. 

Die  \colorbox{yellow}{Mathe-Leistung} der Person $i$ ($i \in \{1,\dots,300\}$) wird mit $x_{i}$
bezeichnet. Der Mittelwert dieser Personengruppe $\bar{x}$ ergibt sich zu \colorbox{yellow}{538.92} Punkten. Die Standardabweichung
dieser Personengruppe $\hat{\sigma}_{X}$ beträgt \colorbox{yellow}{101.636} Punkte.

\vspace{0.5cm}

Die  \colorbox{green!30}{Lese-Leistung} der Person $i$ ($i \in \{1,\dots,300\}$) wird wieder  mit $y_{i}$
bezeichnet. Der Mittelwert dieser Personengruppe $\bar{y}$ ergibt sich zu \colorbox{green!30}{$523.431$} Punkten. Die Standardabweichung dieser Personengruppe $\hat{\sigma}_{Y}$ beträgt 
\colorbox{green!30}{$94.457$} Punkte.
Der \colorbox{red!30}{Korrelationskoeffizient nach Pearson} hat den Wert \colorbox{red!30}{0.889.}
\end{frame}

%\begin{frame}{Das Streudiagramm}
%\begin{figure}[ht]
% 	\centering
% 	      \includegraphics[width=0.55\textwidth]{scatter_line.pdf}
% 	      \caption{Lese- und Mathe-Punkte der ersten 300 Schülerinnen und Schüler.\label{fig2}}
%\end{figure}
%\end{frame}


\begin{frame}{Die Ausgleichsgerade}
\begin{itemize}
\item{Zeichnet man ein Streudiagramm mit dem Merkmal $X$
auf der Abszisse und dem Merkmal $Y$ auf der Ordinate, so ist die
Steigung der Ausgleichsgeraden definiert als:
 \begin{equation}
 \label{def_beta}
 \hat{\beta}:=\hat{\rho}\cdot \bigg(\frac{\hat{\sigma}_{Y}}{\hat{\sigma}_{X}}\bigg)
 \end{equation}
}
 \item{Der Achsenabschnitt $\hat{\alpha}$ berechnet sich zu:
 \begin{equation}
 \label{def_alpha}
 \hat{\alpha}:=\bar{y}-\hat{\beta}\bar{x}
 \end{equation}
 }
\end{itemize}
\begin{block}{Die Ausgleichsgerade}
$$
\hat{y}_{i} = \hat{\alpha} + \hat{\beta} \cdot x_{i} 
$$
\end{block}
\end{frame}

\begin{frame}{Modellierung der mittleren Tendenz}
\begin{itemize}
\item{Im Streudiagramm der Mathe- und Lese-Punkte der ersten $300$ Schülerinnen und Schüler,
erkennt man dass mit wachsenden Mathe-Punkten die Lese-Punkte ansteigen (bzw. umgekehrt).}\pause
\item{Dieser gleichsinnige Zusammenhang spiegelt sich in $r_{X,Y}=0.889$ wieder.}\pause
\item{Die \underline{mittlere Tendenz} wird durch die Ausgleichsgerade beschrieben.}\pause
\item{In der Übung haben wir die Ausgleichsgerade berechnet.
Sie ergibt sich zu
$$
\hat{y}_{i} = 78.283 + 0.826 \cdot x_{i}.
$$
}
\end{itemize}
\end{frame}

%\begin{frame}{Das Streudiagramm}
%\begin{figure}[ht]
% 	\centering
% 	      \includegraphics[width=0.55\textwidth]{scatter_line.pdf}
% 	      \caption{Lese- und Mathe-Punkte der ersten 300 Schülerinnen und Schüler.\label{fig2}}
%\end{figure}
%\end{frame}

\begin{frame}{Modellierung mit Fehlerterm }
\begin{itemize}
\item{Allerdings sieht man an dem Streudiagramm deutlich, 
dass die Ausgleichsgerade,
die wenigsten Schülerinnen und Schüler trifft. }\pause
\item{Die meisten weichen von der Gerade ab.
Ihr Leistungen entsprechen nicht exakt dem mittleren Tendenz.}\pause
\item{Dieser Tatsache versucht man in der Statistik dadurch Rechnung zu tragen, dass man annimmt,
dass der Zusammenhang zwischen $X$ und $Y$ durch die mittlere Tendenz beschreiben wird und
von einem \underline{additiven Fehlerterm} überlagert.}
\end{itemize}
\end{frame}

\begin{frame}{Additiver Fehlerterm}
\begin{block}{Die Ausgleichsgerade  mit Fehlerterm}
\begin{equation}
\label{eq:2}
y_{i} = \alpha + \beta \cdot x_{i} +\epsilon_{i}
\end{equation}
\end{block}
\begin{itemize}
\item{Der ``Begriff'' Fehlerterm kommt daher, dass das Modell einen Fehler macht in
der Beschreibung der Punktwolke. Dieser Fehler hat für den $i$ten Punkt den Wert $(y_{i}-\hat{y}_{i}).$}\pause
\item{Der Betrag $|y_{i}-\hat{y}_{i}|$ beschreibt die individuelle Abweichung des $i$ten Punkt
von der Geraden ohne die Richtung dieser Abweichung.}\pause
\item{Der Fehlerterm $\epsilon_{i}$ ist positiv, wenn die Person $i$ über der Geraden liegt und 
negativ, wenn die Person $i$ unter der Geraden liegt.}\pause
\end{itemize}
\end{frame}
\begin{frame}{Modellebene und  Datenebene I}
\begin{block}{Die Gerade auf der Modellebene}
$$
y_{i} = \alpha + \beta \cdot x_{i} +\epsilon_{i}
$$
\end{block}
\begin{itemize}
\item{Diese Gleichung beschreibt ein ein mathematisches 
Modell.}\pause
\item{Die griechischen Buchstaben heißen Parameter.}\pause
\item{Die griechischen Buchstaben sind Platzhalter für
Zahlen. }\pause
\item{Wenn man diese Zahlen festlegt, dann
legt man das Modell fest. Es ist dann spezifiziert.}\pause
\item{Wenn man ein spezifiziertes Modell hat,
dann kann man Daten simulieren.}\pause
\item{In diesem Sinn ist ein Modell eine Anleitung
zur Datensimulation.}
\end{itemize}
\end{frame}

\begin{frame}{Modellebene und Datenebene II}
\begin{block}{Die Gerade auf der  Datenebene}
$$
\hat{y}_{i} = \hat{\alpha} + \hat{\beta} \cdot x_{i} + \hat{\epsilon}_{i},
$$
mit $\hat{\epsilon}_{i}= y_{i}-\hat{y}_{i}.$
\end{block}
\begin{itemize}
\item{Wenn man zur Festlegung der Parameter
Daten benutzt, dann wendet das Modell empirisch an,
dann nutzt man es auf der   Datenebene.
}\pause
\item{Die Festgelegten Werte werden durch ein Hütchen $\hat{}$
bezeichnet.}\pause
\item{Es gibt für die Festlegung etablierte Methoden,
aber auch eine rein grafische Vorgehensweise entspricht
einem beschreibenden Modell auf der empirischen Ebene.}
\end{itemize}
\end{frame}

\begin{frame}{Erklärte Streuung}
\begin{itemize}
\item{Wenn man ein deskriptives Modell erstellt, hofft man natürlich, dass die Ausgleichsgerade
die Punkte so gut beschreibt, dass der Modellfehler eine untergeordnete Rolle spielt.}\pause
\item{Das Modell zur Beschreibung der Punktwolke zu nutzen, ist nur dann
sinnvoll, wenn ein großer Anteil der Variabilität durch die Gerade beschreiben wird.}\pause
\item{Die Streuungszerlegung beruht darauf die gesamte Variabilität des Merkmals
$Y$ in zwei Teile zu zerlegen:}\pause
\begin{itemize}
\item{Die erklärte Streuung, sie wird durch die Gerade erklärt.}\pause
\item{Die Reststreuung, sie wird nicht durch die Gerade erklärt.}
\end{itemize}
\end{itemize}
\end{frame}

\begin{frame}{Abhängiges Merkmal, unabhängiges Merkmal und Regression}
\begin{itemize}
\item{Man hat ein Merkmal, dessen Werte man mit Hilfe eines anderen Merkmals
erklären möchte.}
\item{Das Wort \glqq erklären\grqq ist hier nicht kausal zu verstehen. Es ist so zu verstehen,
dass man die beobachtete Streuung im abhängigen Merkmal $Y$ auf die Streuung
im unabhängigen Merkmal $X$ zurückführen kann.}
\item{Man spricht von einer \underline{linearen Einfachregression}}
\begin{itemize}
\item{wenn es nur ein unabhängiges Merkmal $X$ gibt}
\item{und man von einer mittleren Tendenz ausgeht, die sich gut durch eine Gerade beschreiben lässt.}
\end{itemize}
\end{itemize}
\end{frame}

\begin{frame}{Linearen Einfachregression}
\begin{itemize}
\item{Man sagt, dass die gesamte Variabilität von $Y$ auf $X$ zurück geführt werden kann,
wenn alle Punkte genau auf einer Geraden liegen.}
%\item{Der Grund: Wenn $X$ keine Streuung hat, hat auch $Y$ keine Streuung.}
\item{Das Wort Regression bedeutet Rückführung.}
\item{Stellen wir uns einen exakten Zusammenhang vor: Wir haben zwei Messverfahren $X$
und $Y.$ Das Verfahren $Y$ liegt immer um $15\%$ über $X.$}
\item{Wenn wir mit Verfahren $X$ gemessen haben, brauchen wir die Messung mit $Y$
nicht mehr zu machen.}
\item{Die Variabilität in $Y$ kann auch die Variabilität in $X$ zurückgeführt werden.}
\end{itemize}
\end{frame}

\begin{frame}{Die Variabilität von $Y$ auf die Variabilität von $X$ zurückführen: Beispiel Taxigeld}
\begin{itemize}
\item{Wir nehmen als unabhängige Variable $X$ die Länge einer Fahrt mit dem Taxi in km.}
\item{Wir nehmen an die Fahrt mit dem Taxi kostet von Person $i$ kostet \colorbox{yellow}{$y_{i}=3+5 \cdot x_{i} $} Euro.}
\item{Hier liegen alle Punkte, die wir beobachten exakt auf einer Geraden, weil der Zusammenhang
exakt gilt. Man spricht auch von einem deterministischen Zusammenhang.}
\item{Wenn wir $x_{i} \equiv c$ als konstant vorgeben, bedeutet das inhaltlich, dass alle Fahrgäste genau $c$ km mit dem Taxi fahren.}
\item{In diesem Fall hat $X$ keine Streuung und dadurch hat auch $Y$ keine Streuung. Denn die Streuung in $Y$ ergibt 
sich aus der Streuung in $X.$}
\end{itemize}
\end{frame}

\begin{frame}{Körpergröße und Wortschatz von Kindern}
Im Aufgabenblatt $5$ ist die folgende Tabelle gegeben:
 \begin{table}[h]
 \centering 
\begin{tabular}{|r|r|r|r|r|r|}
  \hline
  Kind   $i$     &   $1$    & $2$   & $3$   & $4$   & $5$ \\ \hline
  Körpergröße  $x_{i}$ & $130$ & $112$ & $108$ & $114$ & $136$  \\ \hline
  Wortschatz  $y_{i}$ & $37$& $30$ & $20$&  $28$ & $35$ \\ \hline
 % Alter  $z_{i}$ & $12$& $7$ & $6$&  $7$ & $13$ \\ \hline
  \end{tabular}
 \caption{Die Körpergröße  $X$ und der Wortschatz $Y$ in cm gemessen von $5$
 zufällig ausgewählten Kindern. \label{tab1}}
 \end{table}  
 \begin{itemize}
 \item{An diesem kleinen Datensatz kann man sich die Streuungszerlegung
 gut vor Augen führen.}\pause
 \item{Der Wortschatz der Kinder $Y$ soll auf die Körpergröße zurückgeführt 
 werden.}\pause
 \item{In der Übung hatten wir besprochen, dass 
 die Drittvariable Alter den Zusammenhang inhaltlich erklärt.}
 \end{itemize}
\end{frame}

\subsection{ Streuungszerlegung und Bestimmtheitsmaß}



% \begin{frame}{Modellfehler $y_{i}-\hat{y}_{i}$ grafisch}
% \begin{figure}[ht]
% 	\centering
% 	      \includegraphics[width=0.65\textwidth]{scatter_kinder.pdf}
% 	\end{figure}
% \end{frame}
 

\begin{frame}{Die Streuungszerlegung}
\label{sos-folie}
Die in der Statistik sehr bedeutende Formel der Streuungszerlegung lautet:
     \begin{variableblock}{Die Streuungszerlegung}{bg=Orchid!30,fg=black}{bg=Plum!30,fg=black}  	
     \begin{equation}
     \label{sos}
	  \sum_{i=1}^{n}(y_{i}-\bar{y})^{2}= \sum_{i=1}^{n}(\hat{y}_{i}-\bar{y})^{2} + 
	\sum_{i=1}^{n}(y_{i}-\hat{y}_{i})^{2}
	\end{equation}%SST = SSR + SSE
	\end{variableblock}
	\begin{itemize}
	\item{$SST= \sum_{i=1}^{n}(y_{i}-\bar{y})^{2}$ heißt \textbf{S}um of \textbf{S}quars \textbf{T}otal}\pause
	\item{$SSR= \sum_{i=1}^{n}(\hat{y}_{i}-\bar{y})^{2}$ heißt \textbf{S}um of \textbf{S}quars \textbf{R}egression}\pause
	\item{$SSE= \sum_{i=1}^{n}(y_{i}-\hat{y}_{i})^{2}$ heißt \textbf{S}um of \textbf{S}quars \textbf{E}rror}\pause
	\item{Die Gesamtvarianz, berechnet als empirische Varianz ist
	$$
	\frac{1}{n} \sum_{i=1}^{n}(y_{i}-\bar{y})^{2}.
	$$
	}
	\end{itemize}
\end{frame}

\begin{frame}{Bedeutung der Komponenten der Streuungszerlegung}
Die Gesamtvarianz, berechnet als empirische Varianz ist
	$$
	\frac{1}{n} \sum_{i=1}^{n}(y_{i}-\bar{y})^{2}
	$$
\begin{itemize}
\item{$SST= \sum_{i=1}^{n}(y_{i}-\bar{y})^{2}$ entspricht des Gesamtvarianz bis auf den Faktor $\frac{1}{n}.$}
\item{Die Streuungszerlegung besagt inhaltlich, dass sich die Gesamtvariabilität zerlegen lässt in die Variabilität
der Modellpunkte $\hat{y}_{i},$ die auf der Gerade liegen (=SSR) und }
\item{Die Variabilität die dann noch übrig bleibt (=SSE).}
\end{itemize}
\end{frame}

\begin{frame}{Die Streuungszerlegung}
\begin{figure}[ht]
 	\centering
 	      \includegraphics[width=0.75\textwidth]{sos.jpg}%
 	      \caption{Vorsicht: Die Streuungszerlegung gilt nicht komponentenweise,
 	      also nicht für jedes Kind.}
 	\end{figure}
\end{frame}
 %https://stats.stackexchange.com/questions/207841/why-is-sst-sse-ssr-one-variable-linear-regression
 %http://www.dxbydt.com/that-venerable-f-test-2/
 \begin{frame}{Die Streuungszerlegung im Beispiel: \textbf{Zum Nachrechnen}}
 Mit $\bar{y}=30,$ sowie $\hat{\beta}\approx 0.47$ und $\hat{\alpha} \approx -26.40$ berechnet man
  \begin{table}[h]
 \centering 
\begin{tabular}{|r|r|r|r|r|r|}
  \hline% 49   0 100   4  25
  Kind   $i$     &   $1$    & $2$   & $3$   & $4$   & $5$ \\ \hline
   $x_{i}$ & $130$ & $112$ & $108$ & $114$ & $136$  \\ \hline
    $y_{i}$ & $37$& $30$ & $20$&  $28$ & $35$ \\ \hline
    $\hat{y}_{i}$ & $34.70$ & $26.24$ & $24.36$ & $27.18$ & $37.52$  \\ \hline
    $(y_{i}-\bar{y})^{2}$ & $49$& $0$ & $100$&  $4$ & $25$ \\ \hline
    $(\hat{y}_{i}-\bar{y})^{2}$ & 22.09 & 14.14 & 31.81 &  7.95 &56.55 \\ \hline
    $y_{i}-\hat{y}_{i}$ & 2.3 & 3.76 & -4.36 &  0.82 &-2.52 \\ \hline% 2.30  3.76 -4.36  0.82 -2.52
    $(y_{i}-\hat{y}_{i})^{2}$ & 5.29 & 14.14 & 19.01 &  0.67 &6.35 \\ \hline %5.29 14.14 19.01  0.67  6.35
  \end{tabular}
 \caption{Erweiterte Tabelle zur Streuungszerlegung}
 \end{table}  
 \end{frame}
 
%  \begin{frame}{Modellfehler $y_{i}-\hat{y}_{i}$ grafisch}
% \begin{figure}[ht]
% 	\centering
% 	      \includegraphics[width=0.65\textwidth]{scatter_kinder.pdf}
% 	\end{figure}
% \end{frame}
 
 \begin{frame}{Methode der kleinsten Quadrate}
 \begin{itemize}
 \item{Der Schätzer für die Steigung $\hat{\beta}$ und den Achsenabschnitt der Gerade $\hat{\alpha}$
 sind gerade so bestimmt, dass die unerklärte Streuung $SSE =\sum_{i=1}^{n}(\hat{y}_{i}-y_{i})^{2} $
 minimal ist.}\pause
 \item{Man such also $\hat{\alpha}$ und $\hat{\beta},$ so dass
 \begin{equation}
 SSE =\sum_{i=1}^{n}(y_{i}-\hat{y}_{i})^{2}  = \sum_{i=1}^{n}(y_{i}-\hat{\alpha}-\hat{\beta}\cdot x_{i})^{2} %\arrow min
 \end{equation}
 den kleinstmöglichen Wert annimmt.
 }\pause
 \item{Wenn Sie im Streudiagramm auf der letzten Folie eine andere Gerade einzeichnen und die dazugehörigen
 Abweichungen, dann ist es unmöglich, dass Sie eine Gerade finden, deren SSE geringer ist, als das SEE
 der Ausgleichsgeraden.}
 \end{itemize}

 \end{frame}
 
 \begin{frame}{Die Ausgleichsgerade durch Minimierung
 der SSE}
 \begin{itemize}
\item{Die Formeln \eqref{def_beta} und \eqref{def_alpha} von Folie
14 ergeben sich als Lösung dieser Minimierung. Für die Steigung 
$\hat{\beta}$ erhalten wir:
 \begin{equation}
 %\label{def_beta}
 \hat{\beta}=\frac{\sum_{i=1}^{n} (x_{i}-\bar{x}) (y_{i}-\bar{y})}{
 \sum_{i=1}^{n} (x_{i}-\bar{x})^{2}}=\hat{\rho}\cdot \bigg(\frac{\hat{\sigma}_{Y}}{\hat{\sigma}_{X}}\bigg)
 \end{equation}
}
 \item{Für den Achsenabschnitt $\hat{\alpha}$ergibt sich:
 $$
%\label{def_alpha}
 \hat{\alpha}=\bar{y}-\hat{\beta}\bar{x}
$$
 }
\end{itemize}
 
 \end{frame}
 %\subsection{Das Bestimmtheitsmaß}

 \begin{frame}{Anteil der durch die Gerade erklärte Streuung}
 \begin{block}{Ein Maß für die Güte des Modells insgesamt}
 Die zentrale Frage ist: Welcher Anteil der Streuung der 
 $y_{i}$ lässt sich durch die Regression von $Y$ auf $X$ erklären?
 \end{block}
 \begin{description}
 \item{ Die gesamte Streuung der $y_{i}$ lässt sich erfassen über
 $$
 SST = \sum_{i=1}^{n}(y_{i}-\bar{y})^{2}. 
 $$}
 \item{Dieser Term entspricht bis auf den Faktor $\frac{1}{n}$
 der Stichprobenvarianz und wird hier als \underline{Gesamtstreuung} bezeichnet.}
 \end{description}

 \end{frame}
 
 \begin{frame}{Die Gesamtstreuung}
 Die Gesamtstreuung ist: 
  $$
 SST = \sum_{i=1}^{n}(y_{i}-\bar{y}). 
 $$
 Man berechnet sie in drei Schritten:
 \begin{enumerate}
 \item{Zuerst wird für jedes Individuum $i$ die Abweichung vom Mittelwert $\bar{y}$ berechnet.}\pause
 \item{Dann werden diese  Abweichungen  quadriert.}\pause
 \item{Und zuletzt werden die quadrierten Abweichungen aufsummiert.}
 \end{enumerate}
 \end{frame}

 \begin{frame}{Gesamtstreuung und Streuungszerlegung}
 \begin{description}
 \item{In Gleichung \eqref{sos} ist 
 die Streuungszerlegung festgelegt als: 
      $$
\sum_{i=1}^{n}(y_{i}-\bar{y})^{2}= \sum_{i=1}^{n}(\hat{y}_{i}-\bar{y})^{2} + 
	\sum_{i=1}^{n}(y_{i}-\hat{y}_{i})^{2}
	$$}
	\item{Diese Formel wird abgekürzt durch: $SST=SSR+SSE$}
 \end{description}
      \begin{variableblock}{Das Bestimmtheitsmaß $R^{2}$}{bg=Orchid!30,fg=black}{bg=Plum!30,fg=black}  	
     \begin{equation}
     \label{r_squ}
	R^{2}:=\frac{\sum_{i=1}^{n}(\hat{y}_{i}-\bar{y})^{2}}{\sum_{i=1}^{n}(y_{i}-\bar{y})^{2}}  = \frac{SSR}{SST}
	\end{equation}%SST = SSR + SSE
	\end{variableblock}
%161FT S. 
 \end{frame}
 
 \begin{frame}{Eigenschaften des Bestimmtheitsmaß}
 
 \begin{block}{Das Bestimmtheitsmaß}
 Das Bestimmtheitsmaß $R^{2}$ ist definiert als
 \begin{equation}
 R^{2} =\frac{\sum_{i=1}^{n}(\hat{y}_{i}-\bar{y})^{2}}{\sum_{i=1}^{n}(y_{i}-\bar{y})^{2}} =1-\frac{\sum_{i=1}^{n}(y_{i}-\hat{y}_{i})^{2}}{\sum_{i=1}^{n}(y_{i}-\bar{y})^{2}} = 1-\frac{SSE}{SST}.
 \end{equation}
 Es gilt: $0 \leq R^{2} \leq 1,$ $R^{2} =r_{X,Y}^{2}.$
 \end{block}\pause
 \begin{itemize}
 \item{Das Bestimmtheitsmaß ist der Anteil der erklärten Streuung  an der Gesamtstreuung.}\pause
 \item{Es ergibt sich außerdem eine weitere Interpretation für den Korrelationskoeffizient nach Pearson:
 Der quadrierte Korrelationskoeffizient entspricht ebenso wie das Bestimmtheitsmaß 
 dem Anteil der erklärten Streuung an der Gesamtstreuung.}
 \end{itemize}
 \end{frame}
\section{Streuungszerlegung eines metrischen Merkmals in zwei Gruppen}
 
 \subsection{Zwei Gruppen vergleichen}
\begin{frame}{Ein metrisches Merkmal in zwei Gruppen}
\begin{description}
\item{In zahlreichen Situation vergleicht man ein metrisches Merkmal in zwei Gruppen.}\pause
\item{Das $\sim$ Zeichen bedeutet in Abhängigkeit von}
\end{description}\pause
Hier einige Beispiele:
\begin{itemize}
\item{Einkommen $\sim$ Geschlecht}\pause
\item{Pisa-Punkte im Bereich Lesen $\sim$ Geschlecht}\pause
\item{Reaktionszeit $\sim$ Alkohol (ja $\leftrightarrow$ nein)}
\end{itemize}\pause
\begin{block}{Lesen Mädchen und Jungen gleich gut?}
Wir betrachten im Folgenden die Pisa-Punkte im Bereich Lesen $\sim$ Geschlecht
bei den $5001$ 15-jährigen deutschen Schülerinnen und Schülern aus der PISA-Studie $2012.$
\end{block}
\end{frame}

\begin{frame}{Notation}
\begin{itemize}
\item{Ich nutze den Index $j$ um die Gruppen zu differenzieren:
     \[
     j=\left\{\begin{array}{ll} 1, \: \text{``weiblich,''} \\
        2, \:  \text{``männlich''}\end{array}\right. 
  \]

}
\item{Es sei $y_{i,j}$ die Anzahl der Kompetenzpunkte im Bereich Lesen, der $i$-ten Person in Gruppe $j.$
\begin{itemize}
\item{So ist z. B. $y_{3,1}$ die Leseleistung der dritten Schülerin und}\pause
\item{$y_{100,2}$ die Leseleistung des 100. Schülers.}\pause
\end{itemize}
}\pause
\item{Die mittlere Leseleistung der Schülerinnen ist $\bar{y}_{1},$
die mittlere Leseleistung der Jungen ist $\bar{y}_{2}.$ }\pause
\item{Es gibt $n_{1}:=2462$ Schülerinnen und  $n_{2}:=2539$ Schüler im Datensatz.}
\end{itemize}
\end{frame}


\begin{frame}{Die Gesamtstreuung der Leseleistung}
%\begin{block}{Die Gesamtstreuung}
\begin{itemize}
\item{Die Gesamtstreuung bildet man (in drei Schritten) indem man für jedes Individuum die Differenz
vom Mittelwert der gesamten Gruppe bildet, diese Abweichungen quadriert und schließlich aufsummiert.}
\item{Diesen Sachverhalt setzt man in einer Formel um
\begin{equation}
\label{sos_f}
SST=\sum_{i=1}^{n_{1}} (y_{i,1}-\bar{y})^{2} +\sum_{i=1}^{n_{2}} (y_{i,2}-\bar{y})^{2},
 \end{equation}
 mit $n_{1}=2462,$ $n_{2}=2539$ und dem Gesamtmittelwert 
 $$\bar{y}=\frac{1}{n_{1}+n_{2}} \bigg(\sum_{i=1}^{n_{1}} y_{i,1} + \sum_{i=1}^{n_{2}} y_{i,2}\bigg).$$
}
\end{itemize}
%\end{block}
\end{frame}
\subsection{Spezialfall konstanter Gruppen}
% 1    2 
% 2462 2539 
\begin{frame}{Spezialfall: keine Streuung innerhalb der Gruppe}
\begin{itemize}
\item{Ein Extremfall wäre, wenn das Geschlecht die Leseleistung voll bestimmt:\pause
\begin{itemize}
\item{Wir bezeichnen für diesen Fall die Lesepunkte für eine Schülerin mit $a$ und für
einen Schüler mit $b.$}\pause
\item{Es gilt: $$\bar{y}_{1}=\frac{1}{2462}\sum_{i=1}^{2462}y_{i,1} =  \frac{1}{2462}\sum_{i=1}^{2462}a= 
\frac{1}{2462} \cdot 2462 \cdot a =a$$ }
\item{und analog $$\bar{y}_{2} =b. $$}
\end{itemize}}\pause
\item{In diesem Fall gibt es keine Streuung innerhalb der Gruppe der Mädchen
und keine Streuung innerhalb der Gruppe der Jungen.}\pause
\item{Wenn $a=b,$ dann gibt es auch keine Gesamtstreuung, wenn allerdings $a \neq b,$
dann gibt es eine Streuung zwischen den Gruppen.}
\end{itemize}
\end{frame}

\begin{frame}{Gesamtstreuung, bei konstanten Gruppen}
\begin{itemize}
\item{In dem Extremfall konstanter Gruppen, erhalten wir als Gesamtmittelwert
 $$\bar{x}=\frac{1}{n_{1}+n_{2}} \bigg(\sum_{i=1}^{n_{1}} y_{i,1} + \sum_{i=1}^{n_{2}} y_{i,2}\bigg)
 =\frac{1}{n_{1}+n_{2}}(n_{1} \cdot a + n_{2} \cdot b) =$$ $$ = \frac{n_{1} \cdot a + n_{2} \cdot b}{n_{1}+n_{2}}.$$
}\pause
\item{Wenn man diesen Mittelwert in die Formel  der Gesamtstreuung $\eqref{sos_f}$  einsetzt, erhält man
\begin{equation}
\label{sos_t} SST=
n_{1}\bigg( a - \frac{n_{1} \cdot a + n_{2} \cdot b}{n_{1}+n_{2}}\bigg)^{2}+
n_{2}\bigg( b - \frac{n_{1} \cdot a + n_{2} \cdot b}{n_{1}+n_{2}}\bigg)^{2}.
\end{equation}
}
\end{itemize}
\end{frame}
\begin{frame}{Streuung zwischen den Gruppen}
\begin{itemize}
\item{Als Streuung zwischen den Gruppen definiert man die mit der Gruppengröße gewichtete Abweichung
der Gruppenmittelwertes vom Gesamtmittelwert.}\pause
\item{In einer Formel:
\begin{equation}
\label{sos_b} 
SSB := \sum_{j=1}^{2}n_{j}(\bar{y}_{j}-\bar{y})^{2}=n_{1}(\bar{y}_{1}-\bar{y})^{2}+
n_{2}(\bar{y}_{2}-\bar{y})^{2}.
\end{equation}
}\pause
\item{$SSB$ steht für \textbf{S}um of \textbf{S}quares \textbf{B}etween.}
\end{itemize}
\end{frame}

% Rüger II S. 164, 170, 121

\begin{frame}{Streuung zwischen den Gruppen bei konstanten Gruppen}
\begin{itemize}
\item{Im Spezialfall konstanter Gruppen haben wir berechnet: $\bar{y}_{1}=a,$ $\bar{y}_{2}=b$ und 
$\bar{y}=\frac{n_{1} \cdot a + n_{2} \cdot b}{n_{1} + n_{2}}.$}\pause
\item{Wenn man diese Werte  in die Definition der  \textbf{S}um of \textbf{S}quares \textbf{B}etween (Formel \eqref{sos_b}) einsetzt, dann sieht man, dass in diesem Spezialfall gilt \colorbox{yellow}{$SST=SSB.$}}\pause
\item{Im Spezialfall konstanter Gruppen entspricht also die Streuung zwischen den
Gruppen der Gesamtstreuung.}\pause
\item{Das macht inhaltlich Sinn: Denn wenn es innerhalb der Gruppen keine Streuung gibt,
dann ist die einzige Ursache für eine vorhandene Streuung, die Streuung zwischen den Gruppen.}
\end{itemize}
\end{frame}
\subsection{Streuungszerlegung für zwei Gruppen}
\begin{frame}{Der allgemeine Fall: Es gibt auch innerhalb der Gruppen eine Streuung}
\begin{itemize}
\item{In der Realität gibt es innerhalb der Gruppe der Schülerinnen eine Streuung und auch innerhalb
der Gruppe der Schüler.}\pause
\item{Diese Streuung ist gegeben durch folgende Formel:
\begin{equation}
\label{sos_w}
SSW=\sum_{i=1}^{n_{1}} (y_{i,1}-\bar{y}_{1})^{2} +\sum_{i=1}^{n_{2}} (y_{i,2}-\bar{y}_{2})^{2},
 \end{equation}
}\pause
\item{$SSW$ steht für \textbf{S}um of \textbf{S}quares \textbf{W}ithin.}\pause
\item{Man erkennt in Formel \eqref{sos_w}, dass die quadrierten Abweichungen der einzelnen Schülerinnen und Schüler
jeweils um die Mittelwerte ihrer jeweiligen Gruppen maßgeblich sind, für die Streuung innerhalb der Gruppe.}
\end{itemize}
\end{frame}

\begin{frame}{Streuungszerlegung für zwei Gruppen}
Analog zur Streuungszerlegung in der linearen Einfachregression gibt es eine Streuungszerlegung
für ein metrisches Merkmal, dass in zwei Gruppen gemessen wurde,
dieser ist:
      \begin{variableblock}{Streuungszerlegung für zwei Gruppen}{bg=Orchid!30,fg=black}{bg=Plum!30,fg=black}  	
     \begin{equation}
     \label{r_squ}
	\sum_{j=1}^{2}\sum_{i=1}^{n_{j}} (y_{i,j}-\bar{y})^{2} =
	\sum_{j=1}^{2} \sum_{i=1}^{n_{j}} (y_{i,j}-\bar{y}_{j})^{2} + \sum_{j=1}^{2} n_{j} (\bar{y}_{j}-\bar{y})^{2}
	\end{equation}%SST = SSR + SSE
	\end{variableblock}\pause
	\begin{itemize}
	\item{SST=$ \sum_{j=1}^{2}\sum_{i=1}^{n_{j}} (y_{i,j}-\bar{y})^{2}$ heißt \textbf{S}um of \textbf{S}quares \textbf{T}otal.}\pause
	\item{SSB=$\sum_{j=1}^{2} n_{j} (\bar{y}_{j}-\bar{y})^{2}$ heißt \textbf{S}um of \textbf{S}quares \textbf{B}etween.}\pause
	\item{SSW=$ \sum_{j=1}^{2} \sum_{i=1}^{n_{j}} (y_{i,j}-\bar{y}_{j})^{2}$ heißt \textbf{S}um of \textbf{S}quares \textbf{W}ithin. }
	\end{itemize}
\end{frame}

%\begin{frame}{Boxplot: Lesen Jungen und Mädchen gleich gut?}
%\begin{figure}[ht]
% 	\centering
% 	      \includegraphics[width=0.65\textwidth]{lese_boxplot_gender.pdf}
% 	\end{figure}
%\end{frame}

\begin{frame}{Die letzte Übung in diesem Semester}
Die letzte Übung in diesem Semester dient
\begin{itemize}
\item{zur weiteren Klärung der Frage, wer besser liest und}
\item{der Klärung von Fragen im Bezug auf Stoff für die Klausur.}
\item{Bitte gehen Sie die Aufgabenblätter durch und bereiten Sie Fragen vor.}
\item{Das letzte Aufgabenblatt dient der Wiederholung des Stoffes einschließlich der heutigen
Veranstaltung.}
\end{itemize}
\end{frame}

\end{document}


