
\documentclass[a4paper,fontsize=12pt]{scrartcl}
\usepackage[utf8]{inputenc}
\usepackage{hyperref}
\usepackage[ngerman]{babel} 
\usepackage{textcomp} 
\usepackage[margin=1.55cm
  %,showframe% <- only to show the page layout
]{geometry}
\usepackage{rotating}
\usepackage{booktabs} 
\usepackage{lscape}
\usepackage{float}
%opening
\title{ Statistik I}
%\subtitle{Simulation und Auswertung der simulierten Daten in Übung und Tutorium mit Hilfe von \texttt{R}}
%\author{Mariana Nold}
\date{\vspace{-10ex}}
\begin{document}

\maketitle
\begin{enumerate}
\item{\textbf{Motivation: Wozu braucht man Statistik?}
  \begin{itemize}
    \item {Organisation und Ablauf}
    \item {Welche Geschichte erzählen die Daten: Deskriptive Statistik}
   % \item {Vermutung über Signal und Rauschen: Explorative Statistik}
    \item {Ein Signal nachweisen: Induktive Statistik}
    \item Definition: Univariate und bivariate Deskription (Definition)
    \item Rohdaten: Merkmale, Merkmalsträger, Merkmalsausprägungen
    \item {Transparenz ist sehr wichtig!}
  \end{itemize}
  }
  \item \textbf{Qualitative Merkmale}
  \begin{itemize}
     \item Nominales und ordinales Messniveau (qualitative Merkmale)
     \item {Beschreibung von qualitativen Merkmalen durch Grafiken
       \begin{itemize}
       \item{Einfache und gruppierte Säulendiagramme interpretieren}
       \item{Das gruppierte Säulendigramm aus zwei Perspektiven (Rolle der abhängigen bzw.
       unabhängigen Variable)}
       \end{itemize}
     }
     \item Definition: Absolute und relative Häufigkeit
     \item Definition: $2 \times 2$ Kreuztabellen
     \item Definition: Univariate und bivariate Deskription in $2 \times 2$ Kreuztabellen
  \end{itemize}
  
 % \item{Was sind Rohdaten?
   
 % }
  \item{ \textbf{Wie analysiere ich qualitative Merkmale?}
    \begin{itemize}
     \item Wiederholung: $2 \times 2$ Kreuztabellen
     \item {Der $\Phi$-Koeffizient in $2 \times 2$ Kreuztabellen}
     \item{Univariate Deskription und Exploration 
      \begin{itemize}
        \item{Lagemaße: Modus und Median}
        \item{Quantile und kumulative Häufigkeit}
      \end{itemize}
     } % Lagemaße, Kreis- und Balkendiagramm
     \item{Bivariate Deskription und Exploration} % Tabllen, interaktive Grafiken (Mondrian) % Mosaik-Plot
     \begin{itemize}
        \item{Kreuztabellen ($k \times m$), bedingte Häufigkeit} % 
        \item {$\chi^{2}$-Koeffizient}
        \item{Erklärung: Zufallsschwankung (Signal und Rauschen)}
      \end{itemize}
    \end{itemize}
  }
  \item{\textbf{Signal und Rauschen trennen: Was ist Zufall}
	\begin{itemize}
	 \item {Die Bedingte Verteilung eines nominalen Merkmals geg. eines dichotomen Merkmals in der Theorie}
	 \item {Beispiel: Merkmal $X$ ist dichotom ($0$ oder $1$), $Y$ hat vier Ausprägungen, $f_{Y|X}$ durch Festlegung
	 von W.keiten}
	 \item {Theoretische Verteilung $f_{Y|X}$ und beobachtete Daten: Man beobachtet in verschiedenen
	 Stichproben unterschiedliche Verteilungen}
	 \item{Unter der Bedingung $f_{Y|X=0}=f_{Y|X=1}$ welche Werte nimmt der  $\chi^{2}$-Koeffizient an?}
	 \item{Ab wann gehen wir davon aus, dass der beobachtete Werte des  $\chi^{2}$-Koeffizient auf einen
	 Unterschied zwischen $f_{Y|X=0}$und $ f_{Y|X=1}$ hinweist?}
	 \item{$\chi^{2}$-Verteilung, ihre $95\%$-Quantil grafisch Verstehen}
	 \item{Cramer's V und der (korr.) Kontingenzkoeffizient}
	\end{itemize}

  }
  
    \item{\textbf{Univariate Deskription  von  quantitativen Merkmalen}
    \begin{itemize}
    \item{Metrische Merkmale: Intervall- und Verhälnisskala}
     \item{Lagemaße und Streuungsmaße: Modus, Median, Quantile und Mittelwert} % 
        \item{Spannweite, Interquartilsabstand, Varianz und Standardabweichung}
        \item{Graphische Darstellung: Boxplot und Histogramm}
        \item{Kumulative Häufigkeit und emp. Verteilungsfunktion}
        \item{Lageregel: symmetrische und unsymmetrische Verteilungen}
  \end{itemize}
  }
  \item{ \textbf{Die Normalverteilung als univariate Modellverteilung}
       \begin{itemize}
       \item{Historische Einstieg in die Normalverteilung}
            \item{Definition: Modell bzw. Modellverteilung mit der Intention zu Beschreiben}
	    \item{Erwartungswert beschreibt die Lage des Symmetrieachse}
	    \item{Standardabweichung beschreibt Breite der Kurve}
	    \item{Die $68-95-99.7$-Regel}
	    \end{itemize}
        }
     \item{\textbf{Multivariate Deskription und Exploration von  quantitativen Merkmalen}
    \begin{itemize}
          \item{Standardisierung}
          \item{Kovarianz und Korrelation nach Pearson bei standardisierten Merkmalen}
          \item{Korrelation nach Pearson allgemein.}
          \item{Was ist ein gleichsinniger bzw. gegensinniger linearer Zusammenhang?}
         \item{Korrelation und zweidimensionale lineare Regression (deskriptiv)} % 
         \item{Graphische Darstellung: Streudiagramm und Regressionsgerade}
    \end{itemize}
  }
  \item{\textbf{Bivariate Analyse von quantitativen und qualitativen Merkmalen}
    \begin{itemize}
      \item{Monotone Zusammenhänge und Korrelation nach Spearman}
      \item{Kandells $\tau_{\alpha}$}
      \item{Vergleich von Boxplots}
      \item{Die durch die Regressionsgerade erklärte Streuung: Was ist ein Modellfehler?
      (rein deskriptiv, einfach Beschreiben, wie gut das Modell ist)}
    \end{itemize}
  }
  
  \item{\textbf{Vergleich von Mittelwerten und das Konzept des Konfidenzintervalls}
    \begin{itemize}
      \item{Vergleich von Mittelwerten in zwei Gruppen}
      \item{Wie groß muss der Unterschied sein, damit man ihn nicht durch Zufallsschwankungen
      erklären kann.}
      \item{Was ist ein Konfidenzintervall (KI) am Beispiel der Mittelwertdifferenz}
       \begin{itemize}
        \item{bei verbundener Messung}
        \item{bei unverbundener Messung}
       \end{itemize}
       \item{Interpretation des KI als Menge aller Nullhypothesen, die durch die Daten nicht abgelehnt werden}
    \end{itemize}
  }
  
  %\item{\textbf{Wiederholung}}
  
\end{enumerate}

% Aufgaben der deskrip. Statistik:

% ist es, die in den Daten einer Stichprobe enthaltene relevante Information in Tabellen, Grafiken und statistischen 
% % Maßzahlen übersichtlich und in einem der Fragestellung angemessenen Format zusammenzufassen.

% In der Drittvariablenkontrolle versucht man
% Effekte zwischen zwei Variablen unter
% Konstanthaltung (Aussschaltung) der
% anderen ins Modell aufgenommenen
% Variablen zu berechnen

\end{document}
